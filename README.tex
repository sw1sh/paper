
   ¦----------------------------------------------¦
   ¦                                              ¦
   ¦       THE POLYGLOSSIA PACKAGE v1.32.0        ¦
   ¦                                              ¦
   ¦     Modern multilingual typesetting          ¦
   ¦        with XeLaTeX and LuaLaTeX             ¦
   ¦                                              ¦
   ¦----------------------------------------------¦

This package provides a complete Babel replacement for users of XeLaTeX and
(at an early stage) LuaLaTeX. This version includes support for 74 different
languages.

Polyglossia makes it possible to automate the following tasks:

* Loading the appropriate hyphenation patterns.
* Setting the script and language tags of the current font (if possible and
  available), using the package fontspec.
* Switching to a font assigned by the user to a particular script or language.
* Adjusting some typographical conventions in function of the current language
  (such as afterindent, frenchindent, spaces before or after punctuation marks,
  etc.).
* Redefining the document strings (like “chapter”, “figure”, “bibliography”).
* Adapting the formatting of dates (for non-gregorian calendars via external
  packages bundled with polyglossia: currently the Hebrew, Islamic and Farsi
  calendars are supported).
* For languages that have their own numeration system, modifying the formatting
  of numbers appropriately.
* Ensuring the proper directionality if the document contains languages
  written from right to left (via the package bidi, available separately).

Several features of Babel that do not make sense in the XeTeX world (like font
encodings, shorthands, etc) are not supported. For German however, several
shorthands do make sense and can be activated with the language option
babelshorthands=true.

Polyglossia is distributed in the traditional way with *.dtx and *.ins files,
and also comes with a TDS-conformant ready-to-unpack zip file.

To install from source (i.e. using polyglossia.dtx), run
xelatex polyglossia.dtx
which will generate all files and produce the pdf documentation all at once.
Alternatively if you have the file polyglossia.ins run
xetex polyglossia.ins
and follow the instructions displayed.

BUGS

Users are encouraged to report bugs on the github tracker:
http://github.com/reutenauer/polyglossia/issues

CHANGES

1.32.0 (15-05-2013)
  Transitional version to support right-to-left languages with LuaTeX.

1.3 (11-05-2013) / 1.31 (10-05-2013)
  * Several bugfixes.
  * Sync with Babel 3.9.
  * Consolidated support for LuaTeX for all languages but the ones using
    South and South-East Asian scripts, and languages written from right
    to left.  Many thanks to Élie Roux for his help.
  * Added support for Tibetan, contributed by Élie Roux (end of lines are experimental).

1.30MM (06-08-2012)
  * Added support for LuaTeX.  Many languages don’t work yet.  Please be
    patient.

1.2.0e (28-04-2012)
  * Fixed a number of outstanding and not very interesting bugs.
  * Added gloss files for Romansh and Friulan, contributed by Claudio
    Beccari.

1.2.0d (12-01-2012)
  * Removed \makeatletter and \makeother from gloss files entirely.

1.2.0cc (12-10-2011) [First update by Arthur Reutenauer]
  * Update to gloss-italian.ldf by Claudio Beccari, incorporating changes
    by Enrico Gregorio.
  * Conclude every gloss file with \makeatother to match the initial
    \makeatletter.  (Not necessary from a technical point of vue, because of one of the changes by Enrico reported below, but I like it better that way :-)
  * Conclude polyglossia.sty with \ExplSyntaxOff to match the initial
    \ExplSyntaxOn.
  * Added gloss file for Kannada, contributed by Aravinda VK and others.
  * Corrections to the gloss-dutch.ldf thanks to Wouter Bolsterlee.
  * Several patches by Enrico Gregorio, fixing long-standing bugs.
    From the git log:
    - Deleted setup for right-to-left writing direction, see http://tug.org/pipermail/xetex/2011-April/020319.html
    - Changed three appearances of \newcommand to \newrobustcmd, as the commands needs to be protected. Bug reported by "kamensky".
    - Corrected \datepolish as suggested by Piotr Kempa
    - Changed \lccode" into \lccode\string", because it might come into action at wrong times when " is active
    - Changed definition of key xpg@setup, as \@tmpfirst and \@tmpsecond were not expanded, causing dependence of \lefthyphenmin and \righthyphenmin to the last loaded language.  Raised by Vadim Rodionov on the XeTeX mailing list.
    - Deleted \bgroup and \egroup tokens from the definition of otherlanguage*; they serve no purpose (we are already inside an environment) and conflict with csquotes. Noticed by P. Lehman.
    - Changed the calls of \input to \xpg@input, which inputs the required file and resets the catcode of @ to the same value as it had before the input. Since each .ldf file starts with \makeatletter, the old behaviour would leave a category 11 @, which is wrong.
    - Added "\csuse{date#2}" to the definition of "otherlanguage*"

1.2.0b (03-10-2011) [Update by Philipp Stephani]
  * Load xkeyval package explicitly since newer versions
    of fontspec don't load it any more

1.2.0a (27-07-2010) [Last update by François Charette]
  * Initialize \fontfamily acc to value of \familydefault
    (Fixes a bug when using polyglossia with beamer)
  * Remove spurious space in def of \dateenglish
  * Add missing English variant "american"
  * Serbian: fix date format and captions (Latin+Cyrillic)
  * Fix \atticnumeral in gloss-greek
  * Small improvements and fixes in documentation

1.2.0 (15-07-2010)
  * Adapted for fontspec 2.0 (will not work with earlier versions!)
  * New implementation of a \PolyglossiaSetup interface
    that simplifies writing gloss-*.ldf files
  * Many internal enhancements and refactoring
    (including a patch by Dirk Ulrich)
  * Improved automatic font setup when \<lang>font is not defined
  * New environment otherlanguage* (env. equivalent of \foreignlanguage
    (Enrico Gregorio)
  * Bugfix to prevent bogus expansion of \{rm,sf,tt}family even in aux files
    (Enrico Gregorio)
  * New gloss files for Armenian, Bengali, Occitan, Bengali, Lao,
    Malayalam, Marathi, Tamil, Telugu, and Turkmen.
  * New auxiliary packages 'devanagaridigits' and 'bengalidigits'
  * hijrical no longer loads bidi and checks for incompatible l3calc
  * Add Babel shorthands for Russian (based on a patch by Vladimir Lomov)
  * Fix \fnum@{table,figure} for Lithuanian
  * Various improvements in gloss-russian (provided by Vladimir Lomov and
    others)
  * Corrected captions for Bahasai, Lithuanian, Russian, Croatian
  * Add option indentfirst=true for Spanish, Croation and other languages
    (NB: indentfirst was previously named frenchindent)
  * New option 'script' for German: Setting script=fraktur modifies the
    captions for typesetting in Fraktur.
  * New command \aemph for Arabic, Farsi, Urdu, etc. to mark emphasis through
    overlining.
  * Package option 'nolocalmarks' is now true by default: to activate it the
    option 'localmarks' must be passed instead.
  * Loading languages à la Babel as package options is no longer possible (this
    feature was deprecated since v1.1.0).

1.1.1 (23-03-2010)
  * Bugfix for French: explicit spaces before/after double punctuation
    signs ("Par exemple : les grands « espaces » du Canada ! ") are
    now replaced by the appropriate non-breaking spaces, as in Babel.
  * Bugfix for font switching mechanism within Latin script
    (pending a complete re-implementation in v1.2):
    the font shape and series are no longer reset when switching language.
  * New macros for non-Western decimal digits
    (instead of fontmappings)
  * New gloss files for Asturian, Lithuanian and Urdu
  * hijrical.sty is now locale-aware: \hijritoday is
    formatted differently in Arabic, Farsi, Urdu, Turkish
    and Bahasa Indonesia.
---NB: the above five items were not part of v1.1.1-rc1 which was made available on github---
  * Enable babelshorthands for Dutch
  * Add missing macro \allowhyphens
  * Add global option babelshorthands
  * Support Catalan geminated l
  * Bugfix when declaring more than one pkg option
  * Protect \reset@font
  * Add missing requirement makecmds
  * Bugfix for smallcaps in captions
  * Typo for ccname in Hebrew
  * Add option numerals to gloss-russian
  * Provide newXeTeXintercharclass when undefined
  * Bugfix for Russian alph
  * Remove superfluous level of {} in def of markright
  * Bugfix for \datecatalan
  * Change hyphenmins for Sanskrit

1.1.0b
   * Modify hyphenmins for Sanskrit (Yves Codet)
   * Bugfixes for Serbian and Bulgarian (Enrico Gregorio)
1.1.0a
   * Bugfix for interchar tokens
1.1.0
   * Use \newXeTeXintercharclass (thanks to Enrico Gregorio)
   * Fixed implementation of shorthands for German (Babel code in file babelsh.def)
   * Arabic (Khaled Hosny):
     - Fix abjad form for 3 and 5 and add option abjadjimnotail
     - bugfix for \arabicnumber
     - make Gregorian calender the default
     - fixed typos in the sample text
   * Turkish (S. Ö. Yıldız):
     - fix white-space before : and !
     - also check if the font specified TRK for language
     - added missing Turkish translation of "Glossary"
   * Suppress nopattern warning for non-hyphenated scripts
   * Changed U+0163 to U+021B for Romanian (Elie Roux)
   * Stylistic fixes and use macro \xpg@option for package options (E. Gregorio)
   * Fix monthnames in Dutch (A. Ledda)
   * Add Brazilian translation for "glossary"
   * Remove spurious space generated by gloss-spanish
   * Fix ldf file for brazilian
   * Various improvements in the code communicated by E. Gregorio:
     - remove superfluous \protect\language
     - change default language from 0 to \l@nohyphenation=255
     - localize lccode handling of apostrophe in French; add it to Italian
   * Fix frenchspacing for vietnamese
   * Other minor bugfixes

1.0.2
   This is mostly a bug fixes release.
   * Captions corrected in Hebrew, Russian and Spanish
   * Removed all \text<lang> wrappers within caption definitions
   * Improved compatibility with Babel
   * New option "babelshorthands" for German
   * New option "Script" for Sanskrit

1.0.1
   * Improved documentation (added sections on font setup and numeration mappings)
   * Improvements and bugfixes for English and German
   * Bugfix in gloss-syriac.ldf (spurious space after \textsyriac{...})
   * Extended the scope of \syriacabjad
   * Added gloss-amharic.ldf (ported from ethiop.ldf in the package ethiop)

1.0
   * Initial release on CTAN

   ¦----------------------------------------------¦
   ¦                                              ¦
   ¦       THE POLYGLOSSIA PACKAGE v1.32.0        ¦
   ¦                                              ¦
   ¦     Modern multilingual typesetting          ¦
   ¦        with XeLaTeX and LuaLaTeX             ¦
   ¦                                              ¦
   ¦----------------------------------------------¦

This package provides a complete Babel replacement for users of XeLaTeX and
(at an early stage) LuaLaTeX. This version includes support for 74 different
languages.

Polyglossia makes it possible to automate the following tasks:

* Loading the appropriate hyphenation patterns.
* Setting the script and language tags of the current font (if possible and
  available), using the package fontspec.
* Switching to a font assigned by the user to a particular script or language.
* Adjusting some typographical conventions in function of the current language
  (such as afterindent, frenchindent, spaces before or after punctuation marks,
  etc.).
* Redefining the document strings (like “chapter”, “figure”, “bibliography”).
* Adapting the formatting of dates (for non-gregorian calendars via external
  packages bundled with polyglossia: currently the Hebrew, Islamic and Farsi
  calendars are supported).
* For languages that have their own numeration system, modifying the formatting
  of numbers appropriately.
* Ensuring the proper directionality if the document contains languages
  written from right to left (via the package bidi, available separately).

Several features of Babel that do not make sense in the XeTeX world (like font
encodings, shorthands, etc) are not supported. For German however, several
shorthands do make sense and can be activated with the language option
babelshorthands=true.

Polyglossia is distributed in the traditional way with *.dtx and *.ins files,
and also comes with a TDS-conformant ready-to-unpack zip file.

To install from source (i.e. using polyglossia.dtx), run
xelatex polyglossia.dtx
which will generate all files and produce the pdf documentation all at once.
Alternatively if you have the file polyglossia.ins run
xetex polyglossia.ins
and follow the instructions displayed.

BUGS

Users are encouraged to report bugs on the github tracker:
http://github.com/reutenauer/polyglossia/issues

CHANGES

1.32.0 (15-05-2013)
  Transitional version to support right-to-left languages with LuaTeX.

1.3 (11-05-2013) / 1.31 (10-05-2013)
  * Several bugfixes.
  * Sync with Babel 3.9.
  * Consolidated support for LuaTeX for all languages but the ones using
    South and South-East Asian scripts, and languages written from right
    to left.  Many thanks to Élie Roux for his help.
  * Added support for Tibetan, contributed by Élie Roux (end of lines are experimental).

1.30MM (06-08-2012)
  * Added support for LuaTeX.  Many languages don’t work yet.  Please be
    patient.

1.2.0e (28-04-2012)
  * Fixed a number of outstanding and not very interesting bugs.
  * Added gloss files for Romansh and Friulan, contributed by Claudio
    Beccari.

1.2.0d (12-01-2012)
  * Removed \makeatletter and \makeother from gloss files entirely.

1.2.0cc (12-10-2011) [First update by Arthur Reutenauer]
  * Update to gloss-italian.ldf by Claudio Beccari, incorporating changes
    by Enrico Gregorio.
  * Conclude every gloss file with \makeatother to match the initial
    \makeatletter.  (Not necessary from a technical point of vue, because of one of the changes by Enrico reported below, but I like it better that way :-)
  * Conclude polyglossia.sty with \ExplSyntaxOff to match the initial
    \ExplSyntaxOn.
  * Added gloss file for Kannada, contributed by Aravinda VK and others.
  * Corrections to the gloss-dutch.ldf thanks to Wouter Bolsterlee.
  * Several patches by Enrico Gregorio, fixing long-standing bugs.
    From the git log:
    - Deleted setup for right-to-left writing direction, see http://tug.org/pipermail/xetex/2011-April/020319.html
    - Changed three appearances of \newcommand to \newrobustcmd, as the commands needs to be protected. Bug reported by "kamensky".
    - Corrected \datepolish as suggested by Piotr Kempa
    - Changed \lccode" into \lccode\string", because it might come into action at wrong times when " is active
    - Changed definition of key xpg@setup, as \@tmpfirst and \@tmpsecond were not expanded, causing dependence of \lefthyphenmin and \righthyphenmin to the last loaded language.  Raised by Vadim Rodionov on the XeTeX mailing list.
    - Deleted \bgroup and \egroup tokens from the definition of otherlanguage*; they serve no purpose (we are already inside an environment) and conflict with csquotes. Noticed by P. Lehman.
    - Changed the calls of \input to \xpg@input, which inputs the required file and resets the catcode of @ to the same value as it had before the input. Since each .ldf file starts with \makeatletter, the old behaviour would leave a category 11 @, which is wrong.
    - Added "\csuse{date#2}" to the definition of "otherlanguage*"

1.2.0b (03-10-2011) [Update by Philipp Stephani]
  * Load xkeyval package explicitly since newer versions
    of fontspec don't load it any more

1.2.0a (27-07-2010) [Last update by François Charette]
  * Initialize \fontfamily acc to value of \familydefault
    (Fixes a bug when using polyglossia with beamer)
  * Remove spurious space in def of \dateenglish
  * Add missing English variant "american"
  * Serbian: fix date format and captions (Latin+Cyrillic)
  * Fix \atticnumeral in gloss-greek
  * Small improvements and fixes in documentation

1.2.0 (15-07-2010)
  * Adapted for fontspec 2.0 (will not work with earlier versions!)
  * New implementation of a \PolyglossiaSetup interface
    that simplifies writing gloss-*.ldf files
  * Many internal enhancements and refactoring
    (including a patch by Dirk Ulrich)
  * Improved automatic font setup when \<lang>font is not defined
  * New environment otherlanguage* (env. equivalent of \foreignlanguage
    (Enrico Gregorio)
  * Bugfix to prevent bogus expansion of \{rm,sf,tt}family even in aux files
    (Enrico Gregorio)
  * New gloss files for Armenian, Bengali, Occitan, Bengali, Lao,
    Malayalam, Marathi, Tamil, Telugu, and Turkmen.
  * New auxiliary packages 'devanagaridigits' and 'bengalidigits'
  * hijrical no longer loads bidi and checks for incompatible l3calc
  * Add Babel shorthands for Russian (based on a patch by Vladimir Lomov)
  * Fix \fnum@{table,figure} for Lithuanian
  * Various improvements in gloss-russian (provided by Vladimir Lomov and
    others)
  * Corrected captions for Bahasai, Lithuanian, Russian, Croatian
  * Add option indentfirst=true for Spanish, Croation and other languages
    (NB: indentfirst was previously named frenchindent)
  * New option 'script' for German: Setting script=fraktur modifies the
    captions for typesetting in Fraktur.
  * New command \aemph for Arabic, Farsi, Urdu, etc. to mark emphasis through
    overlining.
  * Package option 'nolocalmarks' is now true by default: to activate it the
    option 'localmarks' must be passed instead.
  * Loading languages à la Babel as package options is no longer possible (this
    feature was deprecated since v1.1.0).

1.1.1 (23-03-2010)
  * Bugfix for French: explicit spaces before/after double punctuation
    signs ("Par exemple : les grands « espaces » du Canada ! ") are
    now replaced by the appropriate non-breaking spaces, as in Babel.
  * Bugfix for font switching mechanism within Latin script
    (pending a complete re-implementation in v1.2):
    the font shape and series are no longer reset when switching language.
  * New macros for non-Western decimal digits
    (instead of fontmappings)
  * New gloss files for Asturian, Lithuanian and Urdu
  * hijrical.sty is now locale-aware: \hijritoday is
    formatted differently in Arabic, Farsi, Urdu, Turkish
    and Bahasa Indonesia.
---NB: the above five items were not part of v1.1.1-rc1 which was made available on github---
  * Enable babelshorthands for Dutch
  * Add missing macro \allowhyphens
  * Add global option babelshorthands
  * Support Catalan geminated l
  * Bugfix when declaring more than one pkg option
  * Protect \reset@font
  * Add missing requirement makecmds
  * Bugfix for smallcaps in captions
  * Typo for ccname in Hebrew
  * Add option numerals to gloss-russian
  * Provide newXeTeXintercharclass when undefined
  * Bugfix for Russian alph
  * Remove superfluous level of {} in def of markright
  * Bugfix for \datecatalan
  * Change hyphenmins for Sanskrit

1.1.0b
   * Modify hyphenmins for Sanskrit (Yves Codet)
   * Bugfixes for Serbian and Bulgarian (Enrico Gregorio)
1.1.0a
   * Bugfix for interchar tokens
1.1.0
   * Use \newXeTeXintercharclass (thanks to Enrico Gregorio)
   * Fixed implementation of shorthands for German (Babel code in file babelsh.def)
   * Arabic (Khaled Hosny):
     - Fix abjad form for 3 and 5 and add option abjadjimnotail
     - bugfix for \arabicnumber
     - make Gregorian calender the default
     - fixed typos in the sample text
   * Turkish (S. Ö. Yıldız):
     - fix white-space before : and !
     - also check if the font specified TRK for language
     - added missing Turkish translation of "Glossary"
   * Suppress nopattern warning for non-hyphenated scripts
   * Changed U+0163 to U+021B for Romanian (Elie Roux)
   * Stylistic fixes and use macro \xpg@option for package options (E. Gregorio)
   * Fix monthnames in Dutch (A. Ledda)
   * Add Brazilian translation for "glossary"
   * Remove spurious space generated by gloss-spanish
   * Fix ldf file for brazilian
   * Various improvements in the code communicated by E. Gregorio:
     - remove superfluous \protect\language
     - change default language from 0 to \l@nohyphenation=255
     - localize lccode handling of apostrophe in French; add it to Italian
   * Fix frenchspacing for vietnamese
   * Other minor bugfixes

1.0.2
   This is mostly a bug fixes release.
   * Captions corrected in Hebrew, Russian and Spanish
   * Removed all \text<lang> wrappers within caption definitions
   * Improved compatibility with Babel
   * New option "babelshorthands" for German
   * New option "Script" for Sanskrit

1.0.1
   * Improved documentation (added sections on font setup and numeration mappings)
   * Improvements and bugfixes for English and German
   * Bugfix in gloss-syriac.ldf (spurious space after \textsyriac{...})
   * Extended the scope of \syriacabjad
   * Added gloss-amharic.ldf (ported from ethiop.ldf in the package ethiop)

1.0
   * Initial release on CTAN
