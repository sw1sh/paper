\begin{abstract}
In order to establish an electronic structure of strongly correlated material, the most famous and effective way is to address the Dynamycal mean-field theory.
DMFT maps a lattice problem to a single site problem, with the only approximation of ignoring the spatial correlations between the sites.
To go beyond DMFT to higher-order approximations, one can use an efficient diagrammatic method to describe those correlations - Dual fermion approach.
It makes an exact transition to a dual set of variables and provides a way to treat vertices of an effective single-impurity problem as small parameters.
We then compute some local properties of the one particular model, such as Green's functions within imaginary time representation and corresponding Matsubara space.
Especially we need to know wheather a four-particle vertex is enough to describe the system, and high-order momentas are therefor redundant.
The only way to check this is to derive an expression for the next order vertex and compare them.
\end{abstract}