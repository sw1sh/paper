\documentclass{beamer}

\usepackage[russian, english]{babel}
\usepackage{fontspec}
\setmainfont{Linux Libertine O}
\setsansfont{Linux Libertine O}
\setmonofont{Linux Libertine O}

\usepackage{multido}
\usepackage{fancybox}

\usepackage[backend=bibtex]{biblatex}

\bibliography{bibliography.bib}

\usepackage{tikz}
\usetikzlibrary{decorations, decorations.markings, decorations.pathmorphing, arrows, graphs, graphdrawing, shapes.geometric, snakes, backgrounds,positioning,fit}
\usegdlibrary{trees,force, layered}

\pgfdeclaredecoration{complete sines}{initial}
{
    \state{initial}[
        width=+0pt,
        next state=sine,
        persistent precomputation={\pgfmathsetmacro\matchinglength{
            \pgfdecoratedinputsegmentlength / int(\pgfdecoratedinputsegmentlength/\pgfdecorationsegmentlength)}
            \setlength{\pgfdecorationsegmentlength}{\matchinglength pt}
        }] {}
    \state{sine}[width=\pgfdecorationsegmentlength]{
        \pgfpathsine{\pgfpoint{0.25\pgfdecorationsegmentlength}{0.5\pgfdecorationsegmentamplitude}}
        \pgfpathcosine{\pgfpoint{0.25\pgfdecorationsegmentlength}{-0.5\pgfdecorationsegmentamplitude}}
        \pgfpathsine{\pgfpoint{0.25\pgfdecorationsegmentlength}{-0.5\pgfdecorationsegmentamplitude}}
        \pgfpathcosine{\pgfpoint{0.25\pgfdecorationsegmentlength}{0.5\pgfdecorationsegmentamplitude}}
}
    \state{final}{}
}

\tikzset{
    photon/.style={
        decoration={complete sines, amplitude=0.12cm, segment length=0.2cm},
        decorate    
    },
    fermion/.style={
        decoration={
            markings,
            mark=at position 0.5 with {\node[transform shape, xshift=-0.5mm, fill=black, inner sep=1pt, draw, isosceles triangle]{};}
        },
        postaction=decorate
    },
    gluon/.style={
        decoration={coil, aspect=0.75, mirror, segment length=1.5mm},
        decorate
    }, 
    left/.style={
        bend left=90,
        looseness=1.75
    }
}

\usepackage{ifthen}
\newcommand{\diagram}[1]{
  \begin{tikzpicture}[baseline=(0)]
  \ifthenelse{\equal{#1}{1}}{
  \graph [spring layout, nodes={draw,circle,fill,scale=0.3}, horizontal'=0 to 3]
  {
    0 [coordinate] -- [left, fermion] 1 -- [left, draw] 0,
    1 -- [photon] 2,
    2 -- [left, draw] 3 [coordinate] -- [left, fermion] 2;
  };
  }{
  \graph [spring layout, nodes={draw,circle,fill,scale=0.3}, horizontal'=0 to 1]
  {
    0 -- [photon] 1 -- [left, fermion] 0 -- [left, fermion] 1,
  };
  }
  \end{tikzpicture}
}

\usetheme{Warsaw}


\newcommand{\tikzmark}[1]{\tikz[overlay,remember picture] \node (#1) {};}
\begin{document}
\title{Многочастичные функции Грина коррелированных электронов}
\subtitle{в решеточных структурах}
\author{Николай Мурзин}
\institute{Московский Государственный Университет им.Ломоносова}
\date{23 Мая 2013}

\maketitle

\section{Оглавление}
\begin{frame}
 \tableofcontents
\end{frame}

\section{Введение}
\subsection{Модель Хаббарда}
\begin{frame}[squeez]
  \frametitle{Модель Хаббарда}
  \begin{equation*}
    H = \overbrace{-t \sum_{i,j,\sigma}(c_{i,\sigma}^\dagger c_{j,\sigma} + h.c.)}^{\text{Кинетический член}}
    + \underbrace{U\sum_i n_{i\uparrow}n_{i\downarrow}}_{\text{Кулоновское}\atop\text{взаимодействие}}
    - \overbrace{\mu\sum_{i,\sigma}c_{i\sigma}^\dagger c_{i\sigma}}^{\text{Обмен частицами}}  
  \end{equation*}
  \begin{columns}[C]
    \column{.1\textwidth}
    \column{.4\textwidth}
      \includegraphics[scale=0.2]{img/hubbard.png}
    \column{.5\textwidth}
      \begin{tabular}{l}
	t - Интеграл перескока \\
	\mu - Химический потенциал \\
      \end{tabular}
  \end{columns}
\end{frame}

\subsection{Dynamic Mean Field Theory}
\begin{frame}
 \frametitle{Dynamic Mean Field Theory}
 \begin{block}{DMFT связывает задачу для всей решетки с локальной задачей для одного атома}
   \begin{equation*} S[c,c^*] = \sum_i S_{loc}[c_i,c_i^*] - \sum_{\omega k \sigma}(\Delta_\omega-\epsilon_k)c_{\omega k\sigma}^* c_{\omega k\sigma} \end{equation*} 
 \end{block}
 \begin{center}
   \includegraphics[scale=0.4]{img/dmft.png}
 \end{center}
\end{frame}

\subsection{Устранение нелокальностей}
\begin{frame}
 \frametitle{Учёт нелокальных корреляций}
 \pause
 \begin{block}{Существует несколько подходов}\pause
 \begin{itemize}[<+->]
  \item Взаимодействие электронов на нескольких орбиталях
  \item Локализация кластеров из близлежащих атомов
  \item \tikzmark{s1}{Диаграммная техника в дуальном пространстве}\tikzmark{s2}
 \end{itemize}
 \end{block}
 \tikz[overlay,remember picture]\draw<6->[red,ultra thick,rounded corners] ([yshift=.5em]s1.north west) rectangle (s2.south east);
 \pause
 Разложение идёт по многочастичным вершинам локальной задачи.
\end{frame}

\section{Диаграммы}
\subsection{Лестничные диаграммы}
\begin{frame}
 \frametitle{Лестничные диаграммы}
 Известно, что лестничные диаграммы отвечают за поляризацию материала $\chi$.\pause

 Пока предыдущие научные работы учитывали только лестничные диаграммы с двух-частичными вершинами.
 \begin{center}
  \diagramDMFTfour
 \end{center}
\end{frame}

\subsection{Вершины шестого порядка}
\begin{frame}
 \frametitle{Трёх-частичные вершины}
 Т.к. не существует явного малого параметра разложения, 
  необходимо проанализировать лестничные диаграммы с вершинами шестого порядка.\footnote{\tiny\fullcite{katanin}}\pause
 \begin{center}
  \diagramDMFTsix
 \end{center}
\end{frame}

\section{Проделанная работа}
\subsection{Базовые формулы}
\begin{frame}
 \frametitle{Постановка задачи}
 \pause
 \[ H = -\frac{U}{2} \left(n_{\uparrow}+n_{\downarrow}\right) + U n_{\uparrow} n_{\downarrow} + \frac{U}{2} \] \pause
 \[ G(1\dots n;1'\dots n') = (-1)^n \langle T c_1\dots c_n c_{1'}^\dagger\dots c_{n'}^\dagger\rangle \] \pause
 \[ \Gamma^{(4)} = G(1,2,1',2')+G(1,1')G(2,2')-G(1,2')G(2,1') \] \pause
 \[ \footnotesize\begin{aligned}
      \Gamma^{(6)} & = G(1, 2, 3, 1', 2', 3') \\
      & -2 G(1, 1') G(2, 2') G(3, 3') 
      +2 G(1, 1') G(2, 3') G(3, 2')
      -2 G(1, 2') G(2, 3') G(3, 1') \\ 
      & +2 G(1, 2') G(2, 1') G(3, 3')
      -2 G(1, 3') G(2, 1') G(3, 2')
      +2 G(1, 3') G(2, 2') G(3, 1') \\
      & -G(1, 1') G(2, 3, 2', 3')
      +G(1, 2') G(2, 3, 1', 3')
      -G(1, 3') G(2, 3, 1', 2') \\
      & +G(2, 1') G(1, 3, 2', 3')
      -G(2, 2') G(1, 3, 1', 3')
      +G(2, 3') G(1, 3, 1', 2') \\
      & -G(3, 1') G(1, 2, 2', 3')
      +G(3, 2') G(1, 2, 1', 3')
      -G(3, 3') G(1, 2, 1', 2')
    \end{aligned} 
 \]
\end{frame}

\subsection{Временн\'{о}е представление}
\begin{frame}
 \frametitle{Выведены формулы во временн\'{о}м представлении}
 \pause 
 \[ \Gamma_{\tau_1>\tau_2>\tau_3}^{\multido{}{4}{\uparrow}} = -\frac{1}{4\cosh^2{\frac{U\beta}{4}}}\sinh{(\frac{U}{2}(\tau_1-\tau_2))}\sinh{(\frac{U}{2}\tau_3)} \]
 \pause
 \scriptsize\[ \Gamma_{\tau_1>\tau_2>\tau_3>\tau_4>\tau_5}^{\downarrow\uparrow\uparrow\downarrow\uparrow\uparrow} = 
    -\frac{1}{4\cosh^3{\frac{U\beta}{4}}}\cosh{(\frac{U}{8}(\frac{\beta}{2}-(\tau_1-\tau_4)))}
      \sinh{(\frac{U}{2}(\tau_2-\tau_3))}\sinh{(\frac{U}{2}\tau_5)} \]
 \pause
  \normalsize\dots и аналогичные им для других порядков следования операторов.
\end{frame}

\subsection{Частотное представление}
\begin{frame}
 \frametitle{Мацубаровское представление}
 \pause
 \scriptsize\[ \Gamma_\omega^{(2n)} = \sum_{P_i} \int_0^\beta\int_0^{\tau_{i_1}}\dots\int_0^{\tau_{i_{2n-1}}}\Gamma^{(2n)}(\tau_{i_1},\dots,\tau_{i_{2n}})
  e^{i(\omega_{i_1}\tau_{i_1}+\dots+\omega_{i_{2n}}\tau_{i_{2n}})}d\tau_{i_{2n}}\dots d\tau_{i_1} \]
 \pause
 \[ \beta = \frac{1}{k_B T} \] \pause
 \[ P_i  - \text{Все перестановки временных переменных} \] \pause
 Для параллельных спинов например, двухчастичная вершина имеет очень простой вид
 \[ \gamma^{\uparrow\uparrow\uparrow\uparrow} = \beta \frac{U^2}{4}\frac{\delta_{\omega_1,-\omega_3}-\delta_{\omega_2,-\omega_3}}{\omega_1^2\omega_2^2}
  (\omega_1^2+\frac{U^2}{4})(\omega_2^2+\frac{U^2}{4}) \]
\end{frame}

\subsection{Конечные результаты}
\begin{frame}
 \frametitle{Окончательные результаты для трёх-частичной вершины}\pause
 Был получен не очевидный сходу результат - вершина обращается в нуль, в случае параллельных спинов
 \[ \Gamma^{\multido{}{6}{\uparrow}} = \Gamma^{\multido{}{6}{\downarrow}} = 0 \]\pause
 Приведем разложение по степеням $U$ вершины с одним перевернутым спином, при $\tau_1>\tau_2>\tau_3>\tau_4>\tau_5>\tau_6$ и с попарно равными частотами для двух случаев
 \pause
 \[ \gamma^{\downarrow\uparrow\uparrow\downarrow\uparrow\uparrow}(\omega_1=-\omega_2,\omega_3=-\omega_4,\omega_5=-\omega_6) = O(U^2) \] \pause
 \tiny\begin{align*} & \gamma^{\downarrow\uparrow\uparrow\downarrow\uparrow\uparrow}(\omega_1=-\omega_4,\omega_2 =-\omega_3,\omega_5=-\omega_6) = \\
    & \frac{U^2 \left(\frac{U^2}{4}+\omega _3^2\right){}^2 \left(\frac{U^2}{4}+ \omega _4^2\right){}^2 \left(\frac{U^2}{4}+\omega _5^2\right){}^2 }
      {128 \omega _3^6 \left(\omega _3-\omega _4\right){}^2 \omega _4^6 \omega _5^5 \left(\omega _4+\omega _5\right){}^2 \left(-\omega
   _3+\omega _4+\omega _5\right){}^3} \left(\beta^2 \omega _5 \left(\beta  \omega _5+2 i\right) \omega _4^5 \dots \right) + O(U^4) \end{align*}
\end{frame}

\subsection{Выводы}
\begin{frame}
 \frametitle{Результаты вкратце}
 \pause
 Была написана программа для аналитического вывода выражений для функций Грина и многочастичных вершин, 
  с помощью которой были получены
 \begin{itemize}
  \item Точные формулы для двухчастичной вершины во временном и частотном представлении \pause
  \item Обращение в нуль трёхчастичной вершины с параллельными спинами \pause
  \item Выражения для трёхчастичной вершины с антипараллельными спинами
 \end{itemize}
\end{frame}

\end{document}
