\section{Introduction}
\subsection{Interaction representation}

Given the Hamiltonian $H=H_0+V$ with $H_0$ and $V$ as noninteracting and interacting part respectively,
evolution of states and operators are described by following equations
\[ i\pd{}{t}\ket{\psi_I(t)}=V_I(t)\ket{\psi_I(t)} \]
\[ -i\pd{O_I(t)}{t}=\comm{H_0}{O_I(t)} \]

with states and operators defined as
\[ \ket{\psi_I} = e^{i H_0 t}\ket{\psi_S(t)} \]
\[ O_I(t) = e^{i H_0 t} O_S e^{-i H_0 t} \] 

The evolution of the wavefunction is thus

\[ \ket{\psi_I(t)} = U(t) \ket{\psi_I(0)} \]

so that

\[ i\pd{U(t)}{t} = V(t) U(t) \]

using T-operator we obtain time-ordered exponential

\[ U(t) = T\left[exp\left\{-i\int_{0}^{t}V(t')dt'\right\}\right] \]

\subsection{Finite temperature}
Switching from $t$ to $i\tau$ with $\tau\in\{0,\beta=\frac{1}{k_B T}\}$ we obtain

\[ U(\tau) = T\left[exp\left\{-\int_{0}^{\tau}V(\tau')d\tau'\right\}\right] \]

We can relate the partition function to the evolution operator as follows

\[ Z = Tr\left[e^{-\beta H}\right] = Z_0\left<U(\beta)\right>_0 \]

where 

\[ Z_0 = Tr\left[e^{-\beta H_0}\right] \]

\[ \left<A\right>_0 = \frac{Tr\left[e^{-\beta H_0}A\right]}{Z_0} \]

Now we can write ratio of the interacting to the non-interacting partition function

\[ \frac{Z}{Z_0} = \langle T\ exp\left[-\int_{0}^{\beta}V(\tau)d\tau\right]\rangle \]

\subsection{Perturbation expansion}
In most general form interaction is
\[ V(t) = \tilde{z}(t)c+c^\dagger z(t) \] 

where $z(t)$ and $\tilde{z}(t)$ are the anticommuting forces which ``create'' and ``annihilate'' particles respectively.

Expanding the time-ordered exponential gives us a perturbation series

\[ \frac{Z}{Z_0} = \sum_{n=0}^{\infty}\frac{(-1)^n}{n!}\int_{0}^{\beta}...\int_{0}^{\beta} d\tau_1...d\tau_n 
    \langle T V(\tau_1)...V(\tau_n) \rangle \]

Taking the logarithm of it yields

\[ \log{\frac{Z}{Z_0}} = \sum_{n=1}^{\infty}\frac{(Z/Z_0-1)^n}{n} = \sum_{n=1}^{\infty}
  \frac{\sum_{k=1}^{\infty}\frac{(-1)^k}{k!}\int_{0}^{\beta}...\int_{0}^{\beta} d\tau_1...d\tau_k 
    \langle T V(\tau_1)...V(\tau_k) \rangle}{n} \]

\subsection{Green functions}
We define n-particle Green function as follows

\[ \G(1..n;1'..n') = (-1)^n\langle T c(1)..c(n)c(1')..c(n')\rangle \]

