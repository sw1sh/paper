\section{What was done}
\subsection{Preparations}
Following the dual fermion approach we now consider only the local Hamiltonian of the impurity problem

\begin{equation} H = -\frac{U}{2} \left(n_{\uparrow}+n_{\downarrow}\right) + U n_{\uparrow} n_{\downarrow} + \frac{U}{2} \end{equation}

with impurity states $\ket{0},\ket{\downarrow},\ket{\uparrow},\ket{\downarrow\uparrow}$ and eigenenergies $\frac{U}{2},0,0,\frac{U}{2}$ respectively.

The annihilation operator in this space then can be constructed as follows

\begin{equation} c_\sigma =  	(\ket{0}\bra{\uparrow}-\ket{\downarrow}\bra{\downarrow\uparrow})\delta_{\sigma,\uparrow}
	+	(\ket{0}\bra{\downarrow}+\ket{\downarrow}\bra{\downarrow\uparrow})\delta_{\sigma,\downarrow} \end{equation}

where $\delta_{\sigma_1,\sigma_2}$ - Kronecker's delta symbol.

Since we using an interaction representation, the operator also evolves with time

\begin{equation} c_{\sigma,\tau} = e^{H\tau}c_\sigma e^{-H\tau} \end{equation}

We can now compute the quantum thermodynamic mean of any operator $A$ as follows

\begin{equation} \langle A\rangle = \frac{1}{Z_0}Tr[e^{-H\beta}A] \end{equation}

where $Z_0 = Tr[e^{-H\beta}] = 2(1+e^{-\frac{U\beta}{2}})$.

Given all the above definitions, it's now possible to compute $2n$-particle Green's functions and cumulants of the system

\begin{equation} G(1\dots n;1'\dots n') = (-1)^n \langle c_1\dots c_n c_{1'}^\dagger\dots c_{n'}^\dagger\rangle\end{equation}

\begin{equation} \Gamma^{(2)} = G(1, 1') \end{equation}

\begin{equation} \Gamma^{(4)} = G(1,2,1',2')+G(1,1')G(2,2')-G(1,2')G(2,1')\end{equation}

\begin{equation}\begin{aligned}
\Gamma^{(6)} & = G(1, 2, 3, 1', 2', 3') \\
& -2 G(1, 1') G(2, 2') G(3, 3') 
+2 G(1, 1') G(2, 3') G(3, 2')
-2 G(1, 2') G(2, 3') G(3, 1') \\ 
& +2 G(1, 2') G(2, 1') G(3, 3')
-2 G(1, 3') G(2, 1') G(3, 2')
+2 G(1, 3') G(2, 2') G(3, 1') \\
& -G(1, 1') G(2, 3, 2', 3')
+G(1, 2') G(2, 3, 1', 3')
-G(1, 3') G(2, 3, 1', 2') \\
& +G(2, 1') G(1, 3, 2', 3')
-G(2, 2') G(1, 3, 1', 3')
+G(2, 3') G(1, 3, 1', 2') \\
& -G(3, 1') G(1, 2, 2', 3')
+G(3, 2') G(1, 2, 1', 3')
-G(3, 3') G(1, 2, 1', 2')
\end{aligned}\end{equation}

where variables $1,2\dots n;1',2'\dots n'$ are the shortcuts for corresponding spin and time $(\sigma_i,\tau_i)$
 and also summation over time ordering ($(2n)!$ permutations $\beta>\tau_{i_1}>\tau_{i_2}>\dots\ >\tau_{i_{2n}}$) is implied.

The vertex functions $\gamma^{(2n)}$ are defined as follows

\begin{equation} \gamma^{(2n)} = \frac{\Gamma_\omega^{(2n)}}{G(\omega_1)\dots G(\omega_{2n})} \end{equation}

with $\Gamma_\omega^{(2n)}$ being an imaginary time Fourier transform of $\Gamma^{(2n)}$ ($\Gamma^{(2n)}$ in Matsubara space)

\begin{equation}
  \Gamma_\omega^{(2n)}(\omega_1,\dots,\omega_{2n}) = \int_0^\beta\int_0^{\tau_1}\dots\int_0^{\tau_{2n-1}}\Gamma^{(2n)}(\tau_1,\dots,\tau_{2n})e^{i(\omega_1\tau_1+\dots+\omega_{2n}\tau_{2n})}d\tau_{2n}\dots d\tau_1
\end{equation}

where the integration is performed as time $\tau_i$ increases $\tau_{2n}<\tau_{2n-1}<\dots<\tau_2<\tau_1<\beta$.

\subsection{Calculations}
Let's start with the easiest example and figure out expressions for $G(1,1')$ and its Fourier transform.
Since $G(1,1')\neq0$ only when spins are parallel and depend only on time difference $\tau_1-\tau_{1'}$, we only have one case to consider

\begin{equation}
 G^{\uparrow\downarrow} = G^{\downarrow\uparrow} = 0
\end{equation}

\begin{align*}
 \label{Gamma2}
 \stepcounter{equation}
 G^{\downarrow\downarrow}(\tau_1-\tau_{1'}=\tau) & = G^{\uparrow\uparrow}(\tau) = (-1)^1\left\langle e^{H\tau} c_\uparrow e^{-H\tau} c_\uparrow^\dagger\right\rangle \\
  & = 
  -\frac{1}{Z}\left[
    \bra{0}e^{-H\beta}e^{H\tau}c_\uparrow e^{-H\tau} c_\uparrow^\dagger\ket{0} +
    \bra{\downarrow}e^{-H\beta}e^{H\tau}c_\uparrow e^{-H\tau} c_\uparrow^\dagger\ket{\downarrow}
  \right] \tag{\theequation} \\
  & = -\frac{1}{2(1+e^{-\frac{U\beta}{2}})}\left[ 
    \bra{0}e^{-\frac{U}{2}\beta}e^{\frac{U}{2}\tau}e^{-0\times\tau}\ket{0} +
    \bra{\downarrow}e^{-0\times\beta}e^{0\times\tau}(-1)e^{-\frac{U}{2}\tau}(-1)\ket{\downarrow}
  \right] \\
  & = -\frac{1}{2(1+e^{-\frac{U\beta}{2}})}\left[
    e^{-\frac{U}{2}\beta+\frac{U}{2}\tau} + e^{-\frac{U}{2}\tau}
  \right] = \frac{1}{2\cosh{\frac{U\beta}{4}}}\cosh{(\frac{U}{2}(\frac{\beta}{2}+\tau))}
\end{align*}

The Fourier transform to a Matsubara space \autoref{sec:matsubara} follows
\begin{equation}\begin{aligned}
 G(\omega) & = \int_0^\beta G(\tau) e^{i\omega\tau} d\tau
  = -\frac{1}{2(1+e^{-\frac{U\beta}{2}})}\left[ 
    e^{-\frac{U}{2}\beta}\int_0^\beta  e^{(i\omega+\frac{U}{2})\tau} d\tau + \int_0^\beta  e^{(i\omega-\frac{U}{2})\tau} d\tau
  \right] \\
 & = -\frac{1}{2(1+e^{-\frac{U\beta}{2}})}\left[
    e^{-\frac{U}{2}\beta} \frac{e^{(i\omega+\frac{U}{2})\beta}-1}{i\omega+\frac{U}{2}} + \frac{e^{(i\omega-\frac{U}{2})\beta}-1}{i\omega-\frac{U}{2}}
  \right]
\end{aligned}\end{equation}

We can now use the fact that $\omega = \frac{\pi}{\beta}(2n+1)$,
 so that making a substitution of this definition into exponents of the form $e^{i\omega\beta}$ simplifies them to just $-1$
\begin{equation}\begin{aligned}
 G(\omega) & = -\frac{1}{2(1+e^{-\frac{U\beta}{2}})}\left[
    e^{-\frac{U}{2}\beta} \frac{-e^{\frac{U}{2}\beta}-1}{i\omega+\frac{U}{2}} + \frac{-e^{-\frac{U}{2}\beta}-1}{i\omega-\frac{U}{2}}
  \right] \\
  & = - \frac{1}{2\bcancel{(1+e^{-\frac{U\beta}{2}})}}\left[
    \frac{(i\omega-\cancel{\frac{U}{2}})\bcancel{(1+e^{-\frac{U}{2}\beta})}+(i\omega+\cancel{\frac{U}{2}})\bcancel{(e^{-\frac{U}{2}\beta}+1)}}{\omega^2+\frac{U}{2}^2}
  \right] = -\frac{i\omega}{\omega^2+\frac{U}{2}^2}
\end{aligned}\end{equation}
 
We can also write the full answer in the following form
\begin{equation}
 G^{\sigma_1,\sigma_2}(\omega_1,\omega_2) = -\frac{i\omega_1}{\omega_1^2+\frac{U}{2}^2}\delta_{\sigma_1,-\sigma_2}\delta_{\omega_1,-\omega_2}
\end{equation}

Even though we have the explicit formulas for exact momenta derivation, it would be quite tedious to perform such derivation by hand of course.
For example, in order to calculate $G^{(4)}$ we need to permute $3!=6$ times three independent time variables and then perform
 triple integration on the result to get $G_\omega^{(4)}$.
For this purpose we bring heavy artillery to our aid of the form of the best known symbolical computation tools - Wolfram Mathematica.
The code that does this, consists of the following basic parts:
\begin{itemize}
 \item To represent operators it's either possible to deal directly with matrices or take advantage of \href{http://homepage.cem.itesm.mx/lgomez/quantum/}{Quantum}
package that provides convenient Bra-Ket notation and Quantum Algebra, which makes the code more concise. Therefore the latter was chosen.
 \item Assigning time-ordering attribute to imaginary time variables ($\tau_i$) and the definition of time-ordering operator T that sorts provided operators accordingly,
with appropriate fermionic permutation sign in front.
 \item To speed up the integration part, we define our own simplified version of integral function that can only deal with polynomials, exponentials and their product.
Also it allows us to carry out Kronecker's delta symbols every time an exponential is integrated.
In other words $\int e^{P(\omega)\tau}d\tau$ isn't just $\frac{e^{P(\omega)\tau}}{P(\omega)}$
 but $\frac{e^{P(\omega)\tau}}{P(\omega)}+(\tau-\frac{e^{P(\omega)\tau}}{P(\omega)})\delta_{P(\omega)}$.
Where $P(\omega)$ is some linear function of $\omega_1\dots\omega_{2n}$ that together with energy conservation law $\omega_1+\dots+\omega_{2n}=0$ 
 gives us a way to get the final result in just one run without substituting different frequency combinations each time or taking various limits in the end.
\end{itemize}

First we repeat the result of the Appendix B in \cite{0809.1051} and compute $\gamma^{(4)}$. The following final formulas thus obtained
\begin{equation} \gamma^{\uparrow\uparrow\uparrow\uparrow} = \beta \frac{U^2}{4}\frac{\delta_{\omega_1,-\omega_3}-\delta_{\omega_2,-\omega_3}}{\omega_1^2\omega_2^2}
  (\omega_1^2+\frac{U^2}{4})(\omega_2^2+\frac{U^2}{4}) 
\end{equation}

\begin{equation}\begin{aligned}
 \gamma^{\downarrow\uparrow\downarrow\uparrow} & = 
 U - \frac{U^3}{8}\frac{\omega_1^2+\omega_2^2+\omega_3^2+\omega_4^2}{\omega_1\omega_2\omega_3\omega_4}
   - \frac{3U^5}{16\omega_1\omega_2\omega_3\omega_4} \\
  &- \beta\frac{U^2}{4}\frac{1}{1+e^{\frac{U\beta}{2}}}\frac{2\delta_{\omega_1,-\omega_2}+\delta_{\omega_2,-\omega_3}}{\omega_1^2\omega_3^2}(\omega_1^2+\frac{U^2}{4})(\omega_3^2+\frac{U^2}{4}) \\
  &+ \beta\frac{U^2}{4}\frac{1}{1+e^{-\frac{U\beta}{2}}}\frac{2\delta_{\omega_1,-\omega_3}+\delta_{\omega_2,-\omega_3}}{\omega_1^2\omega_2^2}(\omega_1^2+\frac{U^2}{4})(\omega_2^2+\frac{U^2}{4})   
\end{aligned}\end{equation}

Next we conclude the absence of the $\Gamma^{(6)}$ with all parallel spins
\begin{equation}
 \Gamma^{\multido{}{6}{\uparrow}} = \Gamma^{\multido{}{6}{\downarrow}} = 0
\end{equation}

Then after some simplifications we can obtain the following expression
\footnote{Compare to \eqref{Gamma2} and
\[
 \Gamma^{\multido{}{4}{\uparrow}} = -\frac{3}{2\cosh^2{\frac{U\beta}{4}}}\cosh{(\frac{U}{4}(\beta-2\tau_3))}\sinh{(\frac{U}{2}(\tau_1-\tau_2))}
\]
}
\begin{equation}
\label{Gamma6}
 \Gamma^{\downarrow\uparrow\uparrow\downarrow\uparrow\uparrow} = 
    2\frac{7\cosh{(\frac{U\beta}{2})-8}}{\cosh^3{\frac{U\beta}{4}}}\cosh{(\frac{U}{2}(\frac{\beta}{2}-\tau_5))}
      \sinh{(\frac{U}{2}(\tau_2-\tau_3))}\sinh{(\frac{U}{2}(\tau_1-\tau_4))}
\end{equation}

It depends only on three time differences, let's redefine them as $\tau_1-\tau_4\rightarrow\tau_1,\tau_2-\tau_3\rightarrow \tau_2,\tau_5\rightarrow\tau_3$.
Therefore we can conclude, that Fourier transform of \eqref{Gamma6} will not vanish only if frequencies are pairwise equal to each other.
The exact formula is quite huge and occupies ~5 pages\footnote{Full expression can be found here: \url{http://github.com/sw1sh/gamma6.pdf}},
 but since we interested only in magnitude order of $U$, lets just settle only for the first Taylor coefficient in front of $U^2$ here

\begin{align*}
 \stepcounter{equation}
 \gamma^{\downarrow\uparrow\uparrow\downarrow\uparrow\uparrow} & =
  \frac{U^2(\omega_1^2+\frac{U^2}{4})(\omega_2^2+\frac{U^2}{4})(\omega_3^2+\frac{U^2}{4})}{12\omega_1^5\omega_2^2\omega_3^5}\times( \\
   &\quad6\beta\omega_3^5(\beta\omega_3-i)+6\omega_2^2\omega_3^2(16-7\beta^2\omega_3^2+8i\beta\omega_3+i\beta^3\omega_3^3)+ \tag{\theequation} \\
   &\quad2\omega_2^4(12-3\beta^2\omega_3^2+6i\beta\omega_3+i\beta^3\omega_3^3)+\omega_2\omega_3^3(24-27\beta^2\omega_3^2+24i\beta\omega_3+2i\beta^3\omega_3^3)+ \\
   &\quad3\omega_2^3\omega_3(28-9\beta^2\omega_3^2+14i\beta\omega_3+2i\beta^3\omega_3^3)\\
   & ) + O(U^2)
\end{align*}

Other spin configurations yield the same result, maybe just with the sign difference.

