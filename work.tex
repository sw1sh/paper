\section{What was done}
Even though the dual-fermion approach provides us with the convenient diagrammatic technique, every previous work in this field only included
 four-legged vertices in the so-called ladder diagrams \cite{0812.2456}
\begin{align}
 \diagramDMFTfour
\end{align}

On the other hand there is some evidence exists, that already suggests the need of the six-legged vertex \cite{1301.7546}.

\begin{align}
 \diagramDMFTsix
\end{align}

This work tries to compute the six-legged vertex in both time and frequency representation.
\subsection{Preparations}
Following the dual fermion approach we now consider only the local Hamiltonian of the impurity problem

\begin{equation} H = -\frac{U}{2} \left(n_{\uparrow}+n_{\downarrow}\right) + U n_{\uparrow} n_{\downarrow} + \frac{U}{2} \end{equation}

with impurity states $\ket{0},\ket{\downarrow},\ket{\uparrow},\ket{\downarrow\uparrow}$ and eigenenergies $\frac{U}{2},0,0,\frac{U}{2}$ respectively.

The annihilation operator in this basis then can be constructed as follows

\begin{equation} c_\sigma =  	(\ket{0}\bra{\uparrow}-\ket{\downarrow}\bra{\downarrow\uparrow})\delta_{\sigma,\uparrow}
	+	(\ket{0}\bra{\downarrow}+\ket{\downarrow}\bra{\downarrow\uparrow})\delta_{\sigma,\downarrow} \end{equation}

where $\delta_{\sigma_1,\sigma_2}$ - Kronecker's delta symbol.

Since we using an interaction representation, the operators also evolve with time

\begin{equation} c_{\sigma,\tau} = e^{H\tau}c_\sigma e^{-H\tau} \end{equation}

We can now compute the quantum thermodynamic mean of any operator $A$ as follows

\begin{equation} \langle A\rangle = \frac{1}{Z_0}Tr[e^{-H\beta}A] \end{equation}

where $Z_0 = Tr[e^{-H\beta}] = 2(1+e^{-\frac{U\beta}{2}})$.

Given all the above definitions, it's now possible to compute $2n$-particle Green's functions and vertices of the system

\begin{equation} G(1\dots n;1'\dots n') = (-1)^n \langle c_1\dots c_n c_{1'}^\dagger\dots c_{n'}^\dagger\rangle\end{equation}

\begin{equation} \Gamma^{(2)} = G(1, 1') \end{equation}

\begin{equation} \Gamma^{(4)} = G(1,2,1',2')+G(1,1')G(2,2')-G(1,2')G(2,1') \end{equation}

\begin{equation}\begin{aligned}
\Gamma^{(6)} & = G(1, 2, 3, 1', 2', 3') \\
& -2 G(1, 1') G(2, 2') G(3, 3') 
+2 G(1, 1') G(2, 3') G(3, 2')
-2 G(1, 2') G(2, 3') G(3, 1') \\ 
& +2 G(1, 2') G(2, 1') G(3, 3')
-2 G(1, 3') G(2, 1') G(3, 2')
+2 G(1, 3') G(2, 2') G(3, 1') \\
& -G(1, 1') G(2, 3, 2', 3')
+G(1, 2') G(2, 3, 1', 3')
-G(1, 3') G(2, 3, 1', 2') \\
& +G(2, 1') G(1, 3, 2', 3')
-G(2, 2') G(1, 3, 1', 3')
+G(2, 3') G(1, 3, 1', 2') \\
& -G(3, 1') G(1, 2, 2', 3')
+G(3, 2') G(1, 2, 1', 3')
-G(3, 3') G(1, 2, 1', 2')
\end{aligned}\end{equation}

where we use $1,2\dots n;1',2'\dots n'$ as the shortcuts for the corresponding spin and time variables $(\sigma_i,\tau_i)$.

Fourier transform in Matsubara representation should take into account time ordering
\begin{equation}
  \Gamma_\omega^{(2n)} = \sum_{P_i} \int_0^\beta\int_0^{\tau_{i_1}}\dots\int_0^{\tau_{i_{2n-1}}}\Gamma^{(2n)}(\tau_{i_1},\dots,\tau_{i_{2n}})
  e^{i(\omega_{i_1}\tau_{i_1}+\dots+\omega_{i_{2n}}\tau_{i_{2n}})}d\tau_{i_{2n}}\dots d\tau_{i_1}
\end{equation}

where $P_i$ are the permutations over time ordering $\tau_{i_1}>\tau_{i_2}>\dots\ >\tau_{i_{2n}}$,
 so that integration is performed as time $\tau_i$ increases $\tau_{i_{2n}}<\tau_{2n-1}<\dots<\tau_2<\tau_1<\beta$.

The $\gamma^{(2n)}$-functions used in diagrammatic technique of dual-fermion approach are defined as normalized vertex functions in Matsubara space
\begin{equation} \gamma^{(2n)} = \frac{\Gamma_\omega^{(2n)}}{G(\omega_1)\dots G(\omega_{2n})} \end{equation}
where $G(\omega_i)$ will be derived later.

\subsection{Calculations}
Let's start with the detailed derivation of $G(1,1')$ and its Fourier transform.
Since $G(1,1')\neq0$ only when spins are parallel and depends only on time difference $\tau_1-\tau_{1'}$, we only left with one case to consider

\begin{equation}
 G^{\uparrow\downarrow} = G^{\downarrow\uparrow} = 0
\end{equation}

\begin{align*}
 \label{Gamma2}
 \stepcounter{equation}
 G^{\downarrow\downarrow}(\tau_1-\tau_{1'}=\tau) & = G^{\uparrow\uparrow}(\tau) = (-1)^1\left\langle e^{H\tau} c_\uparrow e^{-H\tau} c_\uparrow^\dagger\right\rangle \\
  & = 
  -\frac{1}{Z}\left[
    \bra{0}e^{-H\beta}e^{H\tau}c_\uparrow e^{-H\tau} c_\uparrow^\dagger\ket{0} +
    \bra{\downarrow}e^{-H\beta}e^{H\tau}c_\uparrow e^{-H\tau} c_\uparrow^\dagger\ket{\downarrow}
  \right] \tag{\theequation} \\
  & = -\frac{1}{2(1+e^{-\frac{U\beta}{2}})}\left[ 
    \bra{0}e^{-\frac{U}{2}\beta}e^{\frac{U}{2}\tau}e^{-0\times\tau}\ket{0} +
    \bra{\downarrow}e^{-0\times\beta}e^{0\times\tau}(-1)e^{-\frac{U}{2}\tau}(-1)\ket{\downarrow}
  \right] \\
  & = -\frac{1}{2(1+e^{-\frac{U\beta}{2}})}\left[
    e^{-\frac{U}{2}\beta+\frac{U}{2}\tau} + e^{-\frac{U}{2}\tau}
  \right] = \frac{1}{2\cosh{\frac{U\beta}{4}}}\cosh{(\frac{U}{2}(\frac{\beta}{2}+\tau))}
\end{align*}

Next follows a Fourier transform of $G(\tau)$ into Matsubara representation (\autoref{sec:matsubara}) 
\begin{equation}\begin{aligned}
 G(\omega) & = \int_0^\beta G(\tau) e^{i\omega\tau} d\tau
  = -\frac{1}{2(1+e^{-\frac{U\beta}{2}})}\left[ 
    e^{-\frac{U}{2}\beta}\int_0^\beta  e^{(i\omega+\frac{U}{2})\tau} d\tau + \int_0^\beta  e^{(i\omega-\frac{U}{2})\tau} d\tau
  \right] \\
 & = -\frac{1}{2(1+e^{-\frac{U\beta}{2}})}\left[
    e^{-\frac{U}{2}\beta} \frac{e^{(i\omega+\frac{U}{2})\beta}-1}{i\omega+\frac{U}{2}} + \frac{e^{(i\omega-\frac{U}{2})\beta}-1}{i\omega-\frac{U}{2}}
  \right]
\end{aligned}\end{equation}

We can now use the fact that $\omega = \frac{\pi}{\beta}(2n+1)$,
 so that substitution of this definition into exponents, simplifies $e^{i\omega\beta}$ to just $-1$
\begin{equation}\begin{aligned}
 G(\omega) & = -\frac{1}{2(1+e^{-\frac{U\beta}{2}})}\left[
    e^{-\frac{U}{2}\beta} \frac{-e^{\frac{U}{2}\beta}-1}{i\omega+\frac{U}{2}} + \frac{-e^{-\frac{U}{2}\beta}-1}{i\omega-\frac{U}{2}}
  \right] \\
  & = - \frac{1}{2\bcancel{(1+e^{-\frac{U\beta}{2}})}}\left[
    \frac{(i\omega-\cancel{\frac{U}{2}})\bcancel{(1+e^{-\frac{U}{2}\beta})}+(i\omega+\cancel{\frac{U}{2}})\bcancel{(e^{-\frac{U}{2}\beta}+1)}}{\omega^2+\frac{U}{2}^2}
  \right] = -\frac{i\omega}{\omega^2+\frac{U}{2}^2}
\end{aligned}\end{equation}
 
We can also write full answer as
\begin{equation}
 G^{\sigma_1\sigma_2}(\omega_1,\omega_2) = -\frac{i\omega_1}{\omega_1^2+\frac{U}{2}^2}\delta_{\sigma_1,-\sigma_2}\delta_{\omega_1,-\omega_2}
\end{equation}

Even though we have the explicit formulas for exact momenta derivation, it would be quite tedious to perform such derivation by hand of course.
For example, in order to calculate $\gamma^{(4)}$ we need to permute three independent time variables  $3!=6$ times, and each time perform triple integration.
And for $\gamma^{(6)}$ it's $5!=120$ permutations and five integrations.
For this purpose we bring heavy artillery to our aid of the form of the best known symbolical computation tools - Wolfram Mathematica.
The code that does this, consists of the following basic parts:
\begin{itemize}
 \item To represent operators it's either possible to deal directly with matrices or take advantage of \href{http://homepage.cem.itesm.mx/lgomez/quantum/}{Quantum}
package that provides convenient Bra-Ket notation and Quantum Algebra, which makes the code more concise. Therefore the latter was chosen.
 \item Assigning time-ordering attribute to imaginary time variables ($\tau_i$) and the definition of time-ordering operator T that sorts provided operators accordingly,
with appropriate fermionic permutation sign in front.
 \item To speed up the integration part, we define our own simplified version of integral function that can only deal with polynomials, exponentials and their product.
Also it allows us to carry out Kronecker's delta symbols every time an exponential is integrated.
In other words $\int e^{P(\omega)\tau}d\tau$ isn't just $\frac{e^{P(\omega)\tau}}{P(\omega)}$
 but $\frac{e^{P(\omega)\tau}}{P(\omega)}+(\tau-\frac{e^{P(\omega)\tau}}{P(\omega)})\delta_{P(\omega)}$.
Where $P(\omega)$ is some linear function of $\omega_1\dots\omega_{2n}$ that together with energy conservation law $\omega_1+\dots+\omega_{2n}=0$ 
 gives us a way to get the final result in just one run without substituting different frequency combinations each time or taking various limits in the end.
\end{itemize}

First we repeat the result of the Appendix B in \cite{0809.1051} and compute $\gamma^{(4)}$. The following final formulas thus obtained
\begin{equation} 
 \label{gamma4}
 \gamma^{\uparrow\uparrow\uparrow\uparrow} = \beta \frac{U^2}{4}\frac{\delta_{\omega_1,-\omega_3}-\delta_{\omega_2,-\omega_3}}{\omega_1^2\omega_2^2}
  (\omega_1^2+\frac{U^2}{4})(\omega_2^2+\frac{U^2}{4}) 
\end{equation}

\begin{equation}\begin{aligned}
 \label{gamma4anti}
 \gamma^{\downarrow\uparrow\downarrow\uparrow} & = 
 U - \frac{U^3}{8}\frac{\omega_1^2+\omega_2^2+\omega_3^2+\omega_4^2}{\omega_1\omega_2\omega_3\omega_4}
   - \frac{3U^5}{16\omega_1\omega_2\omega_3\omega_4} \\
  &- \beta\frac{U^2}{4}\frac{1}{1+e^{\frac{U\beta}{2}}}\frac{2\delta_{\omega_1,-\omega_2}+\delta_{\omega_2,-\omega_3}}{\omega_1^2\omega_3^2}(\omega_1^2+\frac{U^2}{4})(\omega_3^2+\frac{U^2}{4}) \\
  &+ \beta\frac{U^2}{4}\frac{1}{1+e^{-\frac{U\beta}{2}}}\frac{2\delta_{\omega_1,-\omega_3}+\delta_{\omega_2,-\omega_3}}{\omega_1^2\omega_2^2}(\omega_1^2+\frac{U^2}{4})(\omega_2^2+\frac{U^2}{4})   
\end{aligned}\end{equation}

Note the similarities between vertices of 2nd and 3rd order 
\begin{equation}
 \label{Gamma4}
 \Gamma_{\tau_1>\tau_2>\tau_3}^{\multido{}{4}{\uparrow}} = -\frac{1}{4\cosh^2{\frac{U\beta}{4}}}\sinh{(\frac{U}{2}(\tau_1-\tau_2))}\sinh{(\frac{U}{2}\tau_3)}
\end{equation}

\begin{equation}
\label{Gamma6}
 \Gamma_{\tau_1>\tau_2>\tau_3>\tau_4>\tau_5}^{\downarrow\uparrow\uparrow\downarrow\uparrow\uparrow} = 
    -\frac{1}{4\cosh^3{\frac{U\beta}{4}}}\cosh{(\frac{U}{8}(\frac{\beta}{2}-(\tau_1-\tau_4)))}
      \sinh{(\frac{U}{2}(\tau_2-\tau_3))}\sinh{(\frac{U}{2}\tau_5)}
\end{equation}
Similar expressions can easily be computed for the higher-order vertex functions with any spin configurations.

Then we conclude by computation that the $\Gamma^{(6)}$ vanishes when all spins are parallel
\begin{equation}
 \Gamma^{\multido{}{6}{\uparrow}} = \Gamma^{\multido{}{6}{\downarrow}} = 0
\end{equation}

Even it is possible to compute $\gamma^{\downarrow\uparrow\uparrow\downarrow\uparrow\uparrow}$,
 the result is quite huge, and it's impossible to represent it in any human readable form, even the first not vanishing term of Taylor expansion by $U$ is very big.
However we still can get some results by considering only the case $\tau_1>\tau_2>\dots\ >\tau_{2n}$ and only pairwise equal frequencies.
Also by taking into account only first two Taylor coefficient by $U$, we conclude that
 $\gamma^{\downarrow\uparrow\uparrow\downarrow\uparrow\uparrow}(\omega_1=-\omega_2,\omega_3=-\omega_4,\omega_5=-\omega_6) = O(U^2)$,
 i.e. has no terms up to the 4th order at least (because it should consist of only even terms), but
 $\gamma^{\downarrow\uparrow\uparrow\downarrow\uparrow\uparrow}(\omega_1=-\omega_4,\omega_2=-\omega_3,\omega_5=-\omega_6)$ has the 2nd order term and it's equal to

\begin{dmath}[label={gamma6}]
    -\frac{U^2 \left(U^2+4 \omega _3^2\right){}^2 \left(U^2+4 \omega _4^2\right){}^2 \left(U^2+4 \omega _5^2\right){}^2 }{524288 \omega _3^6 \left(\omega _3-\omega _4\right){}^2 \omega _4^6 \omega _5^5 \left(\omega _4+\omega _5\right){}^2 \left(-\omega
   _3+\omega _4+\omega _5\right){}^3}
    \left(-\left(\beta^2 \omega _5 \left(\beta  \omega _5+2 i\right) \omega _4^5+2 \left(\beta ^3 \omega _5^3+2 \beta  \omega _5+16 i\right) \omega
   _4^4+\omega _5 \left(\beta ^3 \omega _5^3-6 i \beta ^2 \omega _5^2-16 \beta  \omega _5+24 i\right) \omega _4^3-4 i \omega _5^2
   \left(\beta ^2 \omega _5^2-3 i \beta  \omega _5+6\right) \omega _4^2-8 \omega _5^3 \left(\beta  \omega _5-i\right) \omega _4+24
   i \omega _5^4\right) \omega _3^7+\left(5 \beta ^2 \omega _5 \left(\beta  \omega _5+2 i\right) \omega _4^6+\left(13 \beta ^3
   \omega _5^3+10 i \beta ^2 \omega _5^2+12 \beta  \omega _5+96 i\right) \omega _4^5+\omega _5 \left(11 \beta ^3 \omega _5^3-22 i
   \beta ^2 \omega _5^2-60 \beta  \omega _5+136 i\right) \omega _4^4+\omega _5^2 \left(3 \beta ^3 \omega _5^3-34 i \beta ^2 \omega
   _5^2-84 \beta  \omega _5-48 i\right) \omega _4^3-4 i \omega _5^3 \left(3 \beta ^2 \omega _5^2-17 i \beta  \omega _5+12\right)
   \omega _4^2+8 \omega _5^4 \left(14 i-3 \beta  \omega _5\right) \omega _4+72 i \omega _5^5\right) \omega _3^6-\left(10 \beta ^2
   \omega _5 \left(\beta  \omega _5+2 i\right) \omega _4^7+\left(32 \beta ^3 \omega _5^3+42 i \beta ^2 \omega _5^2+4 \beta  \omega
   _5+96 i\right) \omega _4^6+\omega _5 \left(37 \beta ^3 \omega _5^3-6 i \beta ^2 \omega _5^2-88 \beta  \omega _5+200 i\right)
   \omega _4^5+2 \omega _5^2 \left(9 \beta ^3 \omega _5^3-35 i \beta ^2 \omega _5^2-76 \beta  \omega _5-12 i\right) \omega
   _4^4+\omega _5^3 \left(3 \beta ^3 \omega _5^3-54 i \beta ^2 \omega _5^2-160 \beta  \omega _5-192 i\right) \omega _4^3-4 i
   \omega _5^4 \left(3 \beta ^2 \omega _5^2-27 i \beta  \omega _5-20\right) \omega _4^2-24 \omega _5^5 \left(\beta  \omega _5-9
   i\right) \omega _4+72 i \omega _5^6\right) \omega _3^5+\left(\omega _4+\omega _5\right) \left(10 \beta ^2 \omega _5 \left(\beta
    \omega _5+2 i\right) \omega _4^7+4 \left(7 \beta ^3 \omega _5^3+12 i \beta ^2 \omega _5^2-6 \beta  \omega _5+8 i\right) \omega
   _4^6+\omega _5 \left(27 \beta ^3 \omega _5^3+16 i \beta ^2 \omega _5^2-88 \beta  \omega _5+24 i\right) \omega _4^5+2 \omega
   _5^2 \left(5 \beta ^3 \omega _5^3-17 i \beta ^2 \omega _5^2-18 \beta  \omega _5-44 i\right) \omega _4^4+\omega _5^3 \left(\beta
   ^3 \omega _5^3-26 i \beta ^2 \omega _5^2-48 \beta  \omega _5-144 i\right) \omega _4^3-4 i \omega _5^4 \left(\beta ^2 \omega
   _5^2-13 i \beta  \omega _5-4\right) \omega _4^2-8 \omega _5^5 \left(\beta  \omega _5-13 i\right) \omega _4+24 i \omega
   _5^6\right) \omega _3^4-\omega _4 \omega _5 \left(\omega _4+\omega _5\right){}^2 \left(5 \beta ^2 \left(\beta  \omega _5+2
   i\right) \omega _4^6+4 \beta  \left(3 \beta ^2 \omega _5^2+8 i \beta  \omega _5-9\right) \omega _4^5+\left(9 \beta ^3 \omega
   _5^3+24 i \beta ^2 \omega _5^2-92 \beta  \omega _5-120 i\right) \omega _4^4+2 \omega _5 \left(\beta ^3 \omega _5^3-i \beta ^2
   \omega _5^2-2 \beta  \omega _5-28 i\right) \omega _4^3+4 \beta  \omega _5^3 \left(4-i \beta  \omega _5\right) \omega _4^2-8
   \omega _5^3 \left(\beta  \omega _5+4 i\right) \omega _4+16 i \omega _5^4\right) \omega _3^3+\omega _4^2 \omega _5 \left(\omega
   _4+\omega _5\right){}^2 \left(\beta ^2 \left(\beta  \omega _5+2 i\right) \omega _4^6+\beta  \left(3 \beta ^2 \omega _5^2+14 i
   \beta  \omega _5-20\right) \omega _4^5+\left(3 \beta ^3 \omega _5^3+24 i \beta ^2 \omega _5^2-112 \beta  \omega _5-168 i\right)
   \omega _4^4+\omega _5 \left(\beta ^3 \omega _5^3+14 i \beta ^2 \omega _5^2-120 \beta  \omega _5-160 i\right) \omega _4^3+2 i
   \omega _5^2 \left(\beta ^2 \omega _5^2+10 i \beta  \omega _5+60\right) \omega _4^2+8 \omega _5^3 \left(\beta  \omega _5+4
   i\right) \omega _4-16 i \omega _5^4\right) \omega _3^2-2 i \omega _4^3 \omega _5 \left(\omega _4+\omega _5\right){}^3
   \left(\beta  \left(\beta  \omega _5+2 i\right) \omega _4^4+2 \left(\beta ^2 \omega _5^2+14 i \beta  \omega _5-26\right) \omega
   _4^3+\omega _5 \left(\beta ^2 \omega _5^2+32 i \beta  \omega _5-24\right) \omega _4^2+2 \omega _5^2 \left(3 i \beta  \omega
   _5+26\right) \omega _4+8 \omega _5^3\right) \omega _3-12 \omega _4^4 \omega _5 \left(\omega _4+\omega _5\right){}^5
   \left(\omega _4 \left(\beta  \omega _5+2 i\right)-2 i \omega _5\right)\right)
\end{dmath}

\subsection{Conclusion}
In conclusion for presentation purposes we provide the following table that summarizes our main results for the two-particle and three-particle vertices
\begin{equation}
\begin{array}{|c|c|c|}
  \hline
  & \Gamma(\tau) & \gamma(\omega) \\ \hline
  \multido{}{4}{\uparrow},\multido{}{4}{\downarrow} & \eqref{Gamma4} & \eqref{gamma4} \\ \hline
  \downarrow\uparrow\downarrow\uparrow & \text{not present} & \eqref{gamma4anti} \\ \hline
  \multido{}{6}{\uparrow},\multido{}{6}{\downarrow} & 0 & 0 \\ \hline
  \downarrow\uparrow\uparrow\downarrow\uparrow\uparrow & \eqref{Gamma6} & \eqref{gamma6} \\ \hline
\end{array}
\end{equation}
%The debates about whether it's necessary or not to include greater than second perturbation terms in the dual-fermion expansion thus shall continue.