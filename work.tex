\section{What was done}
\subsection{Preparations}
Following the dual fermion approach we now consider only the local Hamiltonian of the impurity problem

\begin{equation} H = -\frac{U}{2} \left(n_{\uparrow}+n_{\downarrow}\right) + U n_{\uparrow} n_{\downarrow} + \frac{U}{2} \end{equation}

with impurity states $\ket{0},\ket{\downarrow},\ket{\uparrow},\ket{\downarrow\uparrow}$ and eigenenergies $\frac{U}{2},0,0,\frac{U}{2}$ respectively.

The annihilation operator in this space then can be constructed as follows

\begin{equation} c_\sigma =  	(\ket{0}\bra{\uparrow}-\ket{\downarrow}\bra{\downarrow\uparrow})\delta_{\sigma,\uparrow}
	+	(\ket{0}\bra{\downarrow}+\ket{\downarrow}\bra{\downarrow\uparrow})\delta_{\sigma,\downarrow} \end{equation}

Since we using an interaction representation, the operator also evolves with time

\begin{equation} c_{\sigma,\tau} = e^{H\tau}c_\sigma e^{-H\tau} \end{equation}

We can now compute the quantum thermodynamic mean of any operator $A$ as follow

\begin{equation} \langle A\rangle = \frac{1}{Z_0}Tr[e^{-H\beta}A] \end{equation}

where $Z_0 = Tr[e^{-H\beta}] = 2(1+e^{-\frac{U\beta}{2}})$.

Given this, it's now possible to compute $2n$-particle Green functions

\begin{equation} G(1\dots n;1'\dots n') = (-1)^n \langle c_1\dots c_n c_{1'}^\dagger\dots c_{n'}^\dagger\rangle\end{equation}

Expressions for $\Gamma^{(2n)}$ are the $n$-th cumulants of the system

\begin{equation} \Gamma^{(2)} = G(1, 1') \end{equation}

\begin{equation} \Gamma^{(4)} = G(1,2,1',2')+G(1,1')G(2,2')-G(1,2')G(2,1')\end{equation}

\begin{equation}\begin{aligned}
\Gamma^{(6)} & = G(1, 2, 3, 1', 2', 3') \\
& -2 G(1, 1') G(2, 2') G(3, 3') 
+2 G(1, 1') G(2, 3') G(3, 2')
-2 G(1, 2') G(2, 3') G(3, 1') \\ 
& +2 G(1, 2') G(2, 1') G(3, 3')
-2 G(1, 3') G(2, 1') G(3, 2')
+2 G(1, 3') G(2, 2') G(3, 1') \\
& -G(1, 1') G(2, 3, 2', 3')
+G(1, 2') G(2, 3, 1', 3')
-G(1, 3') G(2, 3, 1', 2') \\
& +G(2, 1') G(1, 3, 2', 3')
-G(2, 2') G(1, 3, 1', 3')
+G(2, 3') G(1, 3, 1', 2') \\
& -G(3, 1') G(1, 2, 2', 3')
+G(3, 2') G(1, 2, 1', 3')
-G(3, 3') G(1, 2, 1', 2')
\end{aligned}\end{equation}

where variables $1,2\dots n;1',2'\dots n'$ are the shortcuts for corresponding spin and time $(\sigma_i,\tau_i)$
 and also summation over time ordering ($(2n)!$ permutations $\beta>\tau_{i_1}>\tau_{i_2}>\dots\ >\tau_{i_{2n}}$) is implied.

The vertex functions $\gamma^{(2n)}$ are defined as follows

\begin{equation} \gamma^{(2n)} = \frac{\Gamma_\omega^{(2n)}}{G(\omega_1)\dots G(\omega_{2n})} \end{equation}

with $\Gamma_\omega^{(2n)}$ being an imaginary time Fourier transform of $\Gamma^{(2n)}$ ($\Gamma^{(2n)}$ in Matsubara space)

\begin{equation}
  \Gamma_\omega^{(2n)}(\omega_1,\dots,\omega_{2n}) = \int_0^\beta\int_0^{\tau_1}\dots\int_0^{\tau_{2n-1}}\Gamma^{(2n)}(\tau_1,\dots,\tau_{2n})e^{i(\omega_1\tau_1+\dots+\omega_{2n}\tau_{2n})}d\tau_{2n}\dots d\tau_1
\end{equation}

where the integration is performed as time $\tau_i$ increases $\tau_{2n}<\tau_{2n-1}<\dots<\tau_2<\tau_1<\beta$.

\subsection{Calculations}

We require energy conservation law of the form $\omega_1+\dots+\omega_{2n}=0$.

\begin{equation} G(\omega) = -\frac{i\omega}{\frac{U}{2}^2+\omega^2},\quad \omega_1=-\omega_2=\omega \end{equation}

\begin{equation} \gamma^{\uparrow\uparrow\uparrow\uparrow} = \beta \frac{U^2}{4}\frac{\delta_{\omega_1,-\omega_3}-\delta_{\omega_2,-\omega_3}}{\omega_1^2\omega_2^2}
  (\omega_1^2+\frac{U^2}{4})(\omega_2^2+\frac{U^2}{4}) 
\end{equation}


\begin{equation}\begin{aligned}
 \gamma^{\downarrow\uparrow\downarrow\uparrow} & = 
 U - \frac{U^3}{8}\frac{\omega_1^2+\omega_2^2+\omega_3^2+\omega_4^2}{\omega_1\omega_2\omega_3\omega_4}
   - \frac{3U^5}{16\omega_1\omega_2\omega_3\omega_4} \\
  &- \beta\frac{U^2}{4}\frac{1}{1+e^{\frac{U\beta}{2}}}\frac{2\delta_{\omega_1,-\omega_2}+\delta_{\omega_2,-\omega_3}}{\omega_1^2\omega_3^2}(\omega_1^2+\frac{U^2}{4})(\omega_3^2+\frac{U^2}{4}) \\
  &+ \beta\frac{U^2}{4}\frac{1}{1+e^{-\frac{U\beta}{2}}}\frac{2\delta_{\omega_1,-\omega_3}+\delta_{\omega_2,-\omega_3}}{\omega_1^2\omega_2^2}(\omega_1^2+\frac{U^2}{4})(\omega_2^2+\frac{U^2}{4})   
\end{aligned}\end{equation}

