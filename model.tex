\section{Theory and Motivation}
\subsection{Strongly correlated systems}
Strongly correlated system is a class of materials, where the behavior of their electrons can't be 
described effectively in terms of non-interacting entities. Therefore models of the electronic
structure of strongly correlated system must include electronic correlation.

For example, semiconductors and most metals can be described as having non-interacting electrons.
This simple approach is valid because the interaction (Coulomb) energy of electrons is much smaller than their kinetic energy.
However there are important systems for which interactions between the particles are not weak,
 and these interactions play a major role in determining the properties of such systems.
Some examples of such "strongly correlated systems" include:
\begin{itemize}
\item
High-temperature superconductors. The transition temperature for these materials is surprisingly high.
The origin of superconductivity is still unclear, but it is commonly believed that it comes mostly from the Coulomb interaction between the electrons,
 rather than the electron-ion interactions that are important for the conventional superconductors.
\item
Magnetic systems. Coulomb interaction between electrons may lead to a variety of spin ordering patterns, including ferromagnetism and antiferromagnetism.
\item
Quantum Hall systems. In the presence of a strong perpendicular magnetic field electrons confined in one or several two-dimensional layers form a new quantum liquid state,
 that may have some unusual properties.
\item
One dimensional electron systems. Electrons carry two important quantum numbers: charge and spin.
When a single electron moves in a vacuum or in a conventional insulator we can observe charge and spin propagating together.
By contrast in one dimensional systems interacting electrons disintegrate into charge and spin solitons that propagate at different velocities.
\end{itemize}
At this point we do not have a good general approach for understanding "strongly correlated systems".
We only know certain aspects of several systems, and understand very poorly some other ones.

\subsection{The Hubbard model}\label{sec:hubbard}
One of the models of a particular interest is the Hubbard model.
The Hubbard model, named after John Hubbard, is the simplest model of interacting particles in a lattice,
 with only two terms in the Hamiltonian: a kinetic term allowing for tunneling (``hopping'') of particles between sites of the lattice 
 and a potential term consisting of an on-site interaction.
So, having a lattice of atoms with electrons, that are almost localized at each site with possibility to hop
only between nearest neighbors and approximation that electrons interact with each other only in a single orbit, 
the Hamiltonian for the Hubbard model can be thus be written as

\begin{equation} H = -t \sum_{i,j,\sigma}(c_{i,\sigma}^\dagger c_{j,\sigma} + h.c.)+ U\sum_i n_{i\uparrow}n_{i\downarrow} - \mu\sum_{i,\sigma}c_{i\sigma}^\dagger c_{i\sigma} \end{equation}

with the corresponding imaginary-time action

\begin{equation} S = \int_0^\beta d\tau \left\{
  -t \sum_{i,j,\sigma}c_{i,\sigma}^\dagger c_{j,\sigma} + U\sum_i n_{i\uparrow}n_{i\downarrow} - \sum_{i,\sigma}c_{i\sigma}^\dagger (\partial_\tau-\mu) c_{i\sigma} \right\} \end{equation}

and its Fourier transform

\begin{equation} S[c,c^*] = \sum_{\omega k \sigma}(\epsilon_k-\mu-i\omega)c_{\omega k \sigma}^* c_{\omega k \sigma} + U\sum_i\int_0^\beta n_{i\uparrow\tau}n_{i\downarrow\tau} d\tau \end{equation}
 
\begin{tabular}{rl}
  here & $\epsilon_k = -2t(\cos{k_x}+\cos{k_y})$ - the bare dispersion law, \\
  & $c,c^*$ - Grassmann numbers, \\
  & $\omega$ - Matsubara frequencies.
\end{tabular}

\subsection{Mean field theory}
A many-body system with interactions is generally very difficult to solve exactly.
The main idea of MFT is to replace all interactions to any one body with an average or effective interaction - the mean field.
This reduces any lattice problem into an effective single-site problem.
Mean field theory can be applied to a number of physical systems, for example Ising model.

In mean-field theory, the mean field appearing in the single-site problem is a time-independent quantity.
However, this isn't always the case: in a variant of mean-field theory called Dynamical Mean Field Theory (DMFT), the mean-field becomes a time-dependent quantity.
DMFT maps a lattice problem onto a single-site problem. In DMFT, the local observable is the local Green's function.
Thus, the self-consistency condition for DMFT is for the impurity Green's function to reproduce the lattice local Green's function through an effective mean-field.

In our case DMFT maps Hubbard model onto the so-called Anderson impurity model.
This model describes the interaction of one site (the impurity) with a ``bath'' of electronic levels through a hybridization function $\Delta(\tau)$.
The corresponding Hamiltonian for the impurity is then

\begin{equation} H_{imp} = \underbrace{\sum_{\nu,\sigma}\epsilon_\nu n_{\nu,\sigma}^{bath}}_{H_{bath}} + 
	     \underbrace{\sum_{\nu,\sigma}\left(V_{\nu}c_{\sigma}^{\dagger}a_{\nu,\sigma}^{bath}+h.c.\right)}_{H_{mix}}+
	     \underbrace{U n_{\uparrow} n_{\downarrow}-\mu \left(n_{\uparrow}+n_{\downarrow}\right)}_{H_{loc}}
\end{equation}

with the following action

\begin{equation} S = \sum_{\omega \sigma}(\Delta_\omega-\mu-i\omega)c_{\omega\sigma}^* c_{\omega\sigma} + U\int_0^\beta n_{\uparrow\tau}n_{\downarrow\tau} d\tau \end{equation}

The hybridization function $\Delta_\omega$ plays the role of a $\it{dynamic}$ mean field.
In other words it is not static compared to the MFT method.

It is obvious that the impurity problem is much simpler than the origiтal lattice one. And an analytical solution can therefore be obtained.
The only property of the impurity problem entering in the DMFT self-consistent equations is the Green's function $g_{\omega,\sigma}$ in Matsubara space.
The DMFT approximation for the Green's function of the lattice problem with $N$ sites, corresponds to the following expression
\begin{equation}
 G_{\omega,\sigma}^{DMFT} = \frac{1}{N}\sum_k\frac{1}{g_{\omega,\sigma}^{-1}+\Delta_{\omega,\sigma}-\epsilon_k}
\end{equation}
where the hybridization function $\Delta$ satisfies the self-consistency condition of DMFT
\begin{equation}
 G_{r=0,\omega,\sigma}^{DMFT} = g_{\omega,\sigma}
\end{equation}

\subsection{Beyond DMFT}
DMFT approach treats the local spin and orbital fluctuations of the correlated electrons in a correct self-consistent way,
 while spatial correlations between sites are ignored.
The non-perturbed DMFT is successful, because a number of the most important correlation effects are indeed related to local fluctuations.
For example, DMFT describes correctly such phenomena, as the local moment formation in itinerant magnets, some aspects of Kondo physics
 and the Mott insulator-to-metal transition on a lattice with a large connectivity in high-dimensional materials.

On the other hand, there are a lot of examples where non-locality of spatial correlations also plays an important role.
DMFT has several extensions, extending the above formalism to multi-orbital, multi-site problems.
It can be extended with multiple orbitals, so that electron-electron interaction term would include terms denoting different orbitals.
Cluster approximations takes into account the short-range non-local fluctuations in real or k-space.
In these methods, correlations are assumed to be localized within a cluster including several lattice sites.

\subsection{Dual fermion approach}
The dual fermion extension of DMFT\cite{0810.3819} operates with a single-site impurity problem and treats spatial non-locality in a diagrammatic way.
We start from the Hubbard model defined above (\autoref{sec:hubbard}).
The goal is to derive the properties of the initial lattice problem via the quantities of the impurity problem.
The lattice action can be represented in the following way

\begin{equation} S[c,c^*] = \sum_i S_{imp}[c_i,c_i^*] - \sum_{\omega k \sigma}(\Delta_\omega-\epsilon_k)c_{\omega k\sigma}^* c_{\omega k\sigma} \end{equation}

Performing a dual transformation to a set of new Grassmann variables $f,f^*$

\begin{equation} e^{A^2 c_\wks^*c_\wks} = \left(\frac{A}{\alpha}\right)^2 \int 
    e^{-\alpha(c_\wks^*f_\wks+f_\wks^*c_\wks)-\alpha^2 A^{-2}f_\wks^*f_\wks}df_\wks^*df_\wks
 \end{equation}

we can obtain an action depending on the new variables $f,f^*$ only

\begin{equation}\begin{aligned}
S[f,f^*] & = -\sum_{\omega k} \ln{\alpha_{\omega\sigma}^{-2}(\Delta_\omega-\epsilon_k)} - \sum_i\ln{z_i^{imp}} + \\ 
	 & + \sum_\wks \alpha_{\omega\sigma}((\Delta_\omega-\epsilon_k)^{-1}+g_\omega) \alpha_{\omega\sigma}f_\wks^*f_\wks+\sum_i V_i 
\end{aligned}\end{equation}

where $z_i^{imp} = \int e^{-S_{imp}[c_i^*,c_i]}\D c_i^*\D c_i$, and the dual potential $V_i \equiv V[f_i^*,f_i]$ is defined from the expression

\begin{equation} \int e^{-S_{loc}[c_i^*,c_i,f_i^*,f_i]}\D c_i^*\D c_i = 
  z_i^{imp}e^{\sum_{\omega\sigma}\alpha_{\omega\sigma}^2g_\omega f_{\omega i \sigma}^*f_{\omega i \sigma} - V[f_i^*,f_i]} \end{equation}

One can see that the $2n$-th coefficient from the Taylor series of $V[f_i^*,f_i]$ is proportional to vertex function $\gamma^{(2n)}$.
Next we establish an exact relation between the Green's function and higher-order momenta of initial and the dual system as Appendix A of \cite{0810.3819} describes.
For example 

\begin{equation}
 G_{\omega,k} = (\Delta_\omega-\epsilon_k)^{-1}\alpha_{\omega\sigma}G_{\omega,k}^{dual}\alpha_{\omega\sigma}(\Delta_\omega-\epsilon_k)^{-1} + (\Delta_\omega-\epsilon_k)^{-1}
\end{equation}

The main idea of switching to the new variables is that, for a properly chosen $\Delta$, correlation properties of the $f^*,f$ system are simpler than for the 
 $c^*,c$ original model. In other words, the magnitude of the nonlinear part in the dual action can be effectively decreased by the proper choice of $\Delta$.
