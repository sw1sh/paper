\section{Theory}
\subsection{Strongly correlated system}
Strongly correlated system is a class of materials, where the behavior of their electrons can't be 
described effectively in terms of non-interacting entities. Therefore models of the electronic
structure of strongly correlated system must include electronic correlation.

\subsection{The Hubbard model}
Having a lattice of atoms with electrons that are almost localized at each site with possibility to hop
only between nearest neighbors and approximation that electrons interact with each other only in a single orbit, 
the Hubbard model can be thus be written

\[ H = -t \sum_{i,j,\sigma}(c_{i,\sigma}^\dagger c_{j,\sigma} + h.c.)+ U\sum_i n_{i\uparrow}n_{i\downarrow} - \mu\sum_{i,\sigma}c_{i\sigma}^\dagger c_{i\sigma} \]

with the corresponding imaginary-time action

\[ S = \int_0^\beta d\tau \left\{
  -t \sum_{i,j,\sigma}c_{i,\sigma}^\dagger c_{j,\sigma} + U\sum_i n_{i\uparrow}n_{i\downarrow} - \sum_{i,\sigma}c_{i\sigma}^\dagger (\partial_\tau-\mu) c_{i\sigma} \right\} \]

and its Fourier transform

\[ S[c,c^*] = \sum_{\omega k \sigma}(\epsilon_k-\mu-i\omega)c_{\omega k \sigma}^* c_{\omega k \sigma} + U\sum_i\int_0^\beta n_{i\uparrow\tau}n_{i\downarrow\tau} d\tau \]
 
\begin{tabular}{rl}
  here & $\epsilon_k = -2t(\cos{k_x}+\cos{k_y})$ - the bare dispersion law, \\
  & $c,c^*$ - Grassmann numbers, \\
  & $\omega$ - Matsubara frequencies.
\end{tabular}

\subsection{Dynamical mean-field theory}
DMFT maps Hubbard model onto the so-called Anderson impurity model.
This model describes the interaction of one site (the impurity) 
with a ``bath'' of electronic levels through a hybridization function. 

\[ H_{imp} = \underbrace{\sum_{\nu,\sigma}\epsilon_\nu n_{\nu,\sigma}^{bath}}_{H_{bath}} + 
	     \underbrace{\sum_{\nu,\sigma}\left(V_{\nu}c_{\sigma}^{\dagger}a_{\nu,\sigma}^{bath}+h.c.\right)}_{H_{mix}}+
	     \underbrace{U n_{\uparrow} n_{\downarrow}-\mu \left(n_{\uparrow}+n_{\downarrow}\right)}_{H_{loc}}
\]

with the following action

\[ S = \sum_{\omega \sigma}(\Delta_\omega-\mu-i\omega)c_{\omega\sigma}^* c_{\omega\sigma} + U\int_0^\beta n_{\uparrow\tau}n_{\downarrow\tau} d\tau \]

The hybridization function $\Delta_\omega$ plays the role of a $\it{dynamic}$ mean field.

\subsection{Dual fermion approach}
The goal is to derive the properties of the initial lattice problem via the quantities
of the impurity problem.
The lattice action can be represented in the following way

\[ S[c,c^*] = \sum_i S_{imp}[c_i,c_i^*] - \sum_{\omega k \sigma}(\Delta_\omega-\epsilon_k)c_{\omega k\sigma}^* c_{\omega k\sigma} \]

Performing a dual transformation to a set of new Grassmann variables $f,f^*$

\[ e^{A^2 c_\wks^*c_\wks} = \left(\frac{A}{\alpha}\right)^2 \int 
    e^{-\alpha(c_\wks^*f_\wks+f_\wks^*c_\wks)-\alpha^2 A^{-2}f_\wks^*f_\wks}df_\wks^*df_\wks
 \]

we can obtain an action depending on the new variables $f,f^*$ only

\[ S[f,f^*] = -\sum_{\omega k} \ln{\alpha_{\omega\sigma}^{-2}(\Delta_\omega-\epsilon_k)} 
    - \sum_i\ln{z_i^{imp}} + \sum_\wks \alpha_{\omega\sigma}((\Delta_\omega-\epsilon_k)^{-1}+g_\omega)
    \alpha_{\omega\sigma}f_\wks^*f_\wks+\sum_i V_i
\]

where $z_i^{imp} = \int e^{-S_{imp}[c_i^*,c_i]}\D c_i^*\D c_i$, and the dual potential $V_i \equiv V[f_i^*,f_i]$
is defined from the expression

\[ \int e^{-S_{loc}[c_i^*,c_i,f_i^*,f_i]}\D c_i^*\D c_i = 
  z_i^{imp}e^{\sum_{\omega\sigma}\alpha_{\omega\sigma}^2g_\omega f_{\omega i \sigma}^*f_{\omega i \sigma} - V[f_i^*,f_i]} \]

One can see that the $n$-th coefficient from the Taylor series of 
$V[f_i^*,f_i]$ is proportional to $\gamma^{(n)}$, starting from $n=4$.