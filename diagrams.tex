\usepackage{tikz}
\usetikzlibrary{decorations, decorations.markings, decorations.pathmorphing, arrows, graphs, graphdrawing, shapes.geometric, backgrounds,positioning,fit}
\usegdlibrary{trees,force, layered}

\pgfdeclaredecoration{complete sines}{initial}
{
    \state{initial}[
        width=+0pt,
        next state=sine,
        persistent precomputation={\pgfmathsetmacro\matchinglength{
            \pgfdecoratedinputsegmentlength / int(\pgfdecoratedinputsegmentlength/\pgfdecorationsegmentlength)}
            \setlength{\pgfdecorationsegmentlength}{\matchinglength pt}
        }] {}
    \state{sine}[width=\pgfdecorationsegmentlength]{
        \pgfpathsine{\pgfpoint{0.25\pgfdecorationsegmentlength}{0.5\pgfdecorationsegmentamplitude}}
        \pgfpathcosine{\pgfpoint{0.25\pgfdecorationsegmentlength}{-0.5\pgfdecorationsegmentamplitude}}
        \pgfpathsine{\pgfpoint{0.25\pgfdecorationsegmentlength}{-0.5\pgfdecorationsegmentamplitude}}
        \pgfpathcosine{\pgfpoint{0.25\pgfdecorationsegmentlength}{0.5\pgfdecorationsegmentamplitude}}
}
    \state{final}{}
}

\tikzset{
    photon/.style={
        decoration={complete sines, amplitude=0.12cm, segment length=0.2cm},
        decorate    
    },
    fermion/.style={
        decoration={
            markings,
            mark=at position 0.5 with {\node[transform shape, xshift=-0.5mm, fill=black, inner sep=1pt, draw, isosceles triangle]{};}
        },
        postaction=decorate
    },
    gluon/.style={
        decoration={coil, aspect=0.75, mirror, segment length=1.5mm},
        decorate
    }, 
    left/.style={
        bend left=90,
        looseness=1.4
    },
    right/.style={
        bend right=90,
        looseness=1.4
    }
}

\usepackage{ifthen}
\newcommand{\diagram}[1]{
  \begin{tikzpicture}[on grid, baseline={([yshift=-3pt]current bounding box)}]
  \ifthenelse{\equal{#1}{1}}{
  \graph [spring electrical layout, nodes={draw,circle,fill,scale=0.1}, horizontal'=0 to 3]
  {
    0 [coordinate] -- [left, fermion] 1 -- [left, draw] 0,
    1 -- [photon] 2,
    2 -- [left, draw] 3 [coordinate] -- [left, fermion] 2;
  };
  }{
  \ifthenelse{\equal{#1}{2}}{
  \graph [spring layout, nodes={draw,circle,fill,scale=0.1}, horizontal'=0 to 1]
  {
    0 -- [photon] 1 -- [left, fermion] 0 -- [left, fermion] 1,
  };
  }{
  \ifthenelse{\equal{#1}{3}}{
  \graph [spring layout, nodes={draw,circle,fill,scale=0.1}, vertical'=0 to 1]
  {
    1 -- [fermion] 0,
  };
  }{
  \ifthenelse{\equal{#1}{4}}{
  \graph [spring layout, nodes={draw,circle,fill,scale=0.1}, vertical'=0 to 2]
  {
    2 -- [fermion] 1 [coordinate] -- [fermion] 0,
    1 -- [photon] 3,
    3 -- [left, fermion] 4 [coordinate] -- [left, fermion] 3;
  };
  }{
  \tikzset{every node/.style={{draw,circle,fill,scale=0.2}}};
  \node (0) {};
  \node (1) [above =of 0] {};
  \node (2) [above right =of 0] {};
  \node (3) [above left =of 2] {};
  \tikzset{every node/.style={scale=1}};
  \graph [use existing nodes]
  {
    0 -- [fermion] 1 -- [photon] 2 -- [fermion] 3,
    1 -- [right, fermion] 2;
  };
  }}}}
  \end{tikzpicture}
}

\newcommand{\diagramfour}[0]{
\begin{tikzpicture}[on grid, baseline={([yshift=-3pt]current bounding box)}]
  \tikzset{every node/.style={{draw,circle,fill,scale=0.2}}};
  \node (0) at (0,0) {};
  \node (1) at (2,0) {};
  \node (2) at (0,1) {};
  \node (3) at (2,1) {};
  \tikzset{every node/.style={scale=1}};
  \graph [use existing nodes, nodes={draw,circle,fill,scale=0.1}, horizontal'=0 to 1]
  {
    0 -- [double distance=1pt, fermion] 1,
    2 -- [double distance=1pt, fermion] 3;
  };
\end{tikzpicture}
+
\begin{tikzpicture}[on grid, baseline={([yshift=-3pt]current bounding box)}]
  \tikzset{every node/.style={{draw,circle,fill,scale=0.2}}};
  \node (0) at (0,0) {};
  \node (1) at (2,0) {};
  \node (2) at (0,1) {};
  \node (3) at (2,1) {};
  \node [coordinate] (c) at  (1,.5) {};
  \node [coordinate] (l) at (.9,.45) {};
  \node [coordinate] (r) at (1.1,.55) {};
  \tikzset{every node/.style={scale=1}};
  \graph [use existing nodes, nodes={draw,circle,fill,scale=0.1}, horizontal'=0 to 1]
  {
    0 -- [double distance=1pt, fermion] l -- [double distance=1pt, left,draw, looseness=1] r -- [double distance=1pt, draw] 3,
    2 -- [double distance=1pt, fermion] c -- [double distance=1pt, draw] 1;
  };
\end{tikzpicture}
}