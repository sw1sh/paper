\usepackage{tikz}
\usetikzlibrary{decorations, decorations.markings, decorations.pathmorphing, arrows, graphs, graphdrawing, shapes.geometric, snakes, backgrounds,positioning,fit}
\usegdlibrary{trees,force, layered}

\pgfdeclaredecoration{complete sines}{initial}
{
    \state{initial}[
        width=+0pt,
        next state=sine,
        persistent precomputation={\pgfmathsetmacro\matchinglength{
            \pgfdecoratedinputsegmentlength / int(\pgfdecoratedinputsegmentlength/\pgfdecorationsegmentlength)}
            \setlength{\pgfdecorationsegmentlength}{\matchinglength pt}
        }] {}
    \state{sine}[width=\pgfdecorationsegmentlength]{
        \pgfpathsine{\pgfpoint{0.25\pgfdecorationsegmentlength}{0.5\pgfdecorationsegmentamplitude}}
        \pgfpathcosine{\pgfpoint{0.25\pgfdecorationsegmentlength}{-0.5\pgfdecorationsegmentamplitude}}
        \pgfpathsine{\pgfpoint{0.25\pgfdecorationsegmentlength}{-0.5\pgfdecorationsegmentamplitude}}
        \pgfpathcosine{\pgfpoint{0.25\pgfdecorationsegmentlength}{0.5\pgfdecorationsegmentamplitude}}
}
    \state{final}{}
}

\tikzset{
    photon/.style={
        decoration={complete sines, amplitude=0.12cm, segment length=0.2cm},
        decorate    
    },
    fermion/.style={
        decoration={
            markings,
            mark=at position 0.5 with {\node[transform shape, xshift=-0.5mm, fill=black, inner sep=1pt, draw, isosceles triangle]{};}
        },
        postaction=decorate
    },
    gluon/.style={
        decoration={coil, aspect=0.75, mirror, segment length=1.5mm},
        decorate
    }, 
    left/.style={
        bend left=90,
        looseness=1.75
    }
}

\usepackage{ifthen}
\newcommand{\diagram}[1]{
  \begin{tikzpicture}[baseline=(0)]
  \ifthenelse{\equal{#1}{1}}{
  \graph [spring layout, nodes={draw,circle,fill,scale=0.3}, horizontal'=0 to 3]
  {
    0 [coordinate] -- [left, fermion] 1 -- [left, draw] 0,
    1 -- [photon] 2,
    2 -- [left, draw] 3 [coordinate] -- [left, fermion] 2;
  };
  }{
  \graph [spring layout, nodes={draw,circle,fill,scale=0.3}, horizontal'=0 to 1]
  {
    0 -- [photon] 1 -- [left, fermion] 0 -- [left, fermion] 1,
  };
  }
  \end{tikzpicture}
}